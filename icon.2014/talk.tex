\documentclass[landscape]{uedslides2C}
\usepackage{comment}
\usepackage{url}
\usepackage{amsmath}
\usepackage{graphicx}
\usepackage{color}
\usepackage{colortbl}
\usepackage{epic,ecltree}
%\usepackage{bar}
\usepackage{eclbip}
\usepackage{fancybox}
%\usepackage{pause} % java -jar ~/Code/statmt/bin/pp4p.jar mtsummit09-talk.pdf mtsummit09-talk.view.pdf
\usepackage{pdfpages}
\usepackage{fancyvrb}
\usepackage[absolute]{textpos}
\renewcommand*\ttdefault{txtt} % 20% tighter than courier
%\usetikzlibrary{shapes}
%\usepackage{tikz-qtree}
\usepackage{natbib}
%\usepackage[english,vietnam]{babel}
%\usepackage[utf8]{inputenc}
\usepackage{multirow}
\usepackage{multicol}
\usepackage{rotating}
\usepackage[absolute]{textpos}
\usepackage{framed}
\usepackage{scrextend}
\usepackage{bold-extra}
\usepackage{xstring}
\usepackage{xspace}
\usepackage[export]{adjustbox}

\usepackage{tikz}
\usepackage{tikz-qtree}
\usetikzlibrary{arrows,shapes,positioning}
\tikzstyle{textbase} = [text height=1.5ex,text depth=.25ex]
\tikzstyle{pipestep}=[draw,rounded corners,minimum width=3.4cm,textbase]%
\tikzstyle{model}=[pipestep,fill=blue!20]
\tikzstyle{input}=[pipestep,fill=black!50!white,text=white]
\tikzstyle{processing}=[pipestep,fill=green!20]
\tikzstyle{arr}=[->,arrows={-angle 90},line width=4pt,blue!40!black]
% For every picture that defines or uses external nodes, you'll have to
% apply the 'remember picture' style. To avoid some typing, we'll apply
% the style to all pictures.
\tikzstyle{every picture}+=[remember picture]
% By default all math in TikZ nodes are set in inline mode. Change this to
% displaystyle so that we don't get small fractions.
\everymath{\displaystyle}

\definecolor{lightblue}{rgb}{.8,.8,1}
\definecolor{mediumlightblue}{rgb}{.5,.5,1}
\definecolor{lightyellow}{rgb}{1,1,.5}
\definecolor{lightorange}{rgb}{1,.9,.7}
\definecolor{darkorange}{rgb}{1,.75,.2}
\definecolor{verydarkorange}{rgb}{.5,.3,0}
\definecolor{darkblue}{rgb}{0,0,0.8}
\definecolor{verydarkgreen}{rgb}{0,0.4,0}
\definecolor{darkgreen}{rgb}{0,0.8,0}
\definecolor{lightgreen}{rgb}{.8,1,.8}
\definecolor{lightred}{rgb}{1,.8,.8}
\definecolor{gray}{rgb}{0.9,0.9,0.9}
\definecolor{darkgrey}{rgb}{0.5,0.5,0.5}
\definecolor{verydarkgrey}{rgb}{0.3,0.3,0.3}
\definecolor{purple}{rgb}{0.6,0,0.6}
\definecolor{red}{rgb}{1,0,0}
\definecolor{orange}{rgb}{.8,.6,0}
\definecolor{cyan}{rgb}{0,.6,.6}

\newcommand{\newconcept}[1]{\textcolor{blue}{\bf #1}}
\newcommand{\example}[1]{\textcolor{darkblue}{\rm #1}}
\newcommand{\important}[1]{\textcolor{darkblue}{\em #1}}
\newcommand{\concept}[1]{\textcolor{darkblue}{\em #1}}
\newcommand{\maths}[1]{\textcolor{purple}{#1}}
\newcommand{\reference}[1]{\vspace{-2mm}\begin{flushright}\textcolor{purple}{\tiny [from #1]}\end{flushright}\vspace{-7mm}}

\newcommand{\highlightbox}[6]{\begin{textblock}{#3}(#1,#2) \colorbox{#4}{\textcolor{#5}{\begin{minipage}{#3in} #6 \end{minipage} }} \end{textblock}}
\newcommand{\backgroundbox}[5]{\highlightbox{#1}{#2}{#3}{#5}{black}{\vspace{#4in}\hspace{#3in}}}
\newcommand{\currenttopic}[1]{\colorbox{lightyellow}{\textcolor{black}{\bf #1}}}
\newcommand{\littlecode}[1]{\colorbox{gray}{\textcolor{black}{\small \tt #1}}}

\newcommand{\highlight}[1]{\colorbox{lightyellow}{#1}}
\newcommand{\highlightOrange}[1]{\colorbox{lightorange}{#1}}
\newcommand{\highlightGreen}[1]{\colorbox{lightgreen}{#1}}
\newcommand{\highlightBlue}[1]{\colorbox{lightblue}{#1}}

\newcommand{\fragileAcronym}[1]{\StrLen{#1}[\stringlength]\ifnum\stringlength<3\uppercase{#1}\else\textsc{#1}\fi}
\newcommand{\acronym}[1]{\protect\fragileAcronym{#1}}
\newcommand{\cyk}{\acronym{cyk}\xspace}
\DeclareRobustCommand{\cykplus}{\acronym{cyk}\hspace*{-.2ex}\raisebox{0.2ex}{$\scriptstyle +$}\xspace}

\bibliography{mt,more}

\begin{document}
\title[Machine Translation with Open Source Software]{{\sc \huge Moses}\\[3mm] Machine Translation with Open Source Software}
\author[Hoang, Huck and Koehn]{Hieu Hoang and Matthias Huck}
\date{\vspace{-5mm}December 2014}
\maketitle

%%%%%%%%%%%%%%%%%%%%%%%%%%%%%%%%%%%%%%%%%%%%%%%%%%%%%%%%%%%%%%%%%%%%%%%%%%%%
% Introduction

\slide{Outline}
\vfill

\begin{description}
\item[\small 09:30-10:15 $\;\;$] {\bf Introduction} % TODO: should check whether this schedule is realistic
\item[\small 10:15-11:00 $\;\;$] {\bf Hands-on Session} --- you will need a laptop
\item[\small 11:00-11:30 $\;\;$] Break
\item[\small 11:30-13:00 $\;\;$] {\bf Advanced Topics}
\end{description}
\vfill

Slides downloadable from \\ \\
\littlecode{\normalsize http://www.statmt.org/moses/icon.2014.pdf} % TODO: upload slides
\\
\vfill

%%%%%%%%%%%%%%%%%%%%%%%%%%%%%%%%%%%%%%%%%%%%%%%%%%%%%%%%%%%%%%%%%%%%%%%%%%%%


% \slide{Statistical Machine Translation History}
% \vspace{10mm}
% \begin{description}
% \item[around 1990] $\;$\\[2mm] Pioneering work at IBM, inspired by success in speech recognition
% \item[1990s] $\;$\\[2mm] Dominance of IBM's word-based models, support technologies
% \item[early 2000s] $\;$\\[2mm] Phrase-based models
% \item[late 2000s] $\;$\\[2mm] Tree-based models
% \end{description}

%%%%%%%%%%%%%%%%%%%%%%%%%%%%%%%%%%%%%%%%%%%%%%%%%%%%%%%%%%%%%%%%%%%%%%%%%%%%

% \slide{Moses History}
% \vspace{10mm}
% \begin{description} \itemsep -0.2mm
% \item[2002] $\;$ Pharaoh decoder, precursor to Moses (phrase-based models)
% \item[2005] $\;$ Moses started by Hieu Hoang and Philipp Koehn (factored models)
% \item[2006] $\;$ JHU workshop extends Moses significantly
% \item[2006-2012] $\;$ Funding by EU projects EuroMatrix, EuroMatrixPlus
% \item[2009] $\;$ Tree-based models implemented in Moses
% \item[2012-2015] $\;$ MosesCore project. Full-time staff to maintain and enhance Moses
% \end{description}

%%%%%%%%%%%%%%%%%%%%%%%%%%%%%%%%%%%%%%%%%%%%%%%%%%%%%%%%%%%%%%%%%%%%%%%%%%%%

% \slide{Moses in Academia}
% \vspace{10mm}
% \begin{itemize}
% \item Built by academics, for academics
% \item Reference implementation of state of the art
% \begin{itemize}
% \item researchers develop new methods on top of Moses
% \item developers re-implement published methods
% \item used by other researchers as black box
% \end{itemize}
% \item Baseline to beat
% \begin{itemize}
% \item researchers compare their method against Moses
% \end{itemize}
% \end{itemize}

%%%%%%%%%%%%%%%%%%%%%%%%%%%%%%%%%%%%%%%%%%%%%%%%%%%%%%%%%%%%%%%%%%%%%%%%%%%%

% \slide{Developer Community}
% \begin{itemize} \itemsep -1mm
% \item Main development at University of Edinburgh, but also: \vspace{-3mm}
% \begin{itemize}
% \item Fondazione Bruno Kessler (Italy)
% \item Charles University (Czech Republic)
% \item DFKI (Germany)
% \item RWTH Aachen (Germany)
% \item others \ldots
% \end{itemize}
% \item Code shared on github.com
% \item Main forum: Moses support mailing list
% \item Main event: Machine Translation Marathon\vspace{-3mm}
% \begin{itemize}
% \item annual open source convention
% \item presentation of new open source tools
% \item hands-on work on new open source projects
% \item summer school for statistical machine translation
% \end{itemize}
% \end{itemize}

%%%%%%%%%%%%%%%%%%%%%%%%%%%%%%%%%%%%%%%%%%%%%%%%%%%%%%%%%%%%%%%%%%%%%%%%%%%%

% \slide{Other Open Source MT Systems}
% \begin{itemize} \itemsep -3mm
% \item {\bf Joshua} --- Johns Hopkins University\\
% {\small \tt http://joshua.sourceforge.net/}
% \item {\bf CDec} --- University of Maryland\\
% {\small \tt http://cdec-decoder.org/}
% \item {\bf Jane} --- RWTH Aachen\\
% {\small \tt http://www.hltpr.rwth-aachen.de/jane/}
% \item {\bf Phrasal} --- Stanford University\\
% {\small \tt http://nlp.stanford.edu/phrasal/}
% \item Very similar technology \vspace{-2mm}
% \begin{itemize}
% \item Joshua and Phrasal implemented in Java, others in C++
% \item Joshua supports only tree-based models
% \item Phrasal supports only phrase-based models
% \end{itemize}
% \item Open sourcing tools increasing trend in NLP research
% \end{itemize}

%%%%%%%%%%%%%%%%%%%%%%%%%%%%%%%%%%%%%%%%%%%%%%%%%%%%%%%%%%%%%%%%%%%%%%%%%%%%

% \slide{Moses in Industry}
% \vspace{20mm}
% \begin{itemize}
% \item Distributed with LGPL --- free to use
% \item Competitive with commercial SMT solutions\\ (Google, Microsoft, SDL Language Weaver, \ldots)
% \item But:
% \begin{itemize}
% \item not easy to use
% \item requires significant expertise for optimal performance
% \item integration into existing workflow not straight-forward
% \end{itemize}
% \end{itemize}

%%%%%%%%%%%%%%%%%%%%%%%%%%%%%%%%%%%%%%%%%%%%%%%%%%%%%%%%%%%%%%%%%%%%%%%%%%%%

% \slide{Case Studies}
% \vspace{5mm}
% \begin{description} \itemsep -1mm
% \item[European Commission] ---\\  uses Moses in-house to aid human translators
% \item[Autodesk] --- \\ showed productivity increases in translating manuals when post-editing output from a custom-build Moses system
% \item[Systran] --- \\ developed statistical post-editing using Moses
% \item[Asia Online] --- \\ offers translation technology and services based on Moses
% \item[Many others] \ldots \\ World Trade Organisation, Adobe, Symantec, WIPO, Sybase, Safaba, Bloomberg, Pangeanic, KatanMT, Capita, \ldots
% \end{description}

%%%%%%%%%%%%%%%%%%%%%%%%%%%%%%%%%%%%%%%%%%%%%%%%%%%%%%%%%%%%%%%%%%%%%%%%%%%%

\slide{SMT: Basic Idea}
\vfill
\begin{center}
  \hspace*{-6em}\includegraphics[scale=1.8]{basics.pdf}
\end{center}
\vfill

%%%%%%%%%%%%%%%%%%%%%%%%%%%%%%%%%%%%%%%%%%%%%%%%%%%%%%%%%%%%%%%%%%%%%%%%%%%%

\slide{Different Paradigms (1)}
\vfill
\begin{itemize}
\item {\bf Word-based translation}
  \begin{itemize}
  \item Word-based models translate {\em words} as atomic units
  \end{itemize}
\item {\bf Phrase-based translation}
  \begin{itemize}
  \item Phrase-based models translate {\em phrases} as atomic units
  \item Dominant approach nowadays
  \item Advantages:
  \begin{itemize}
  \item larger atomic units capture more context in translation
  \item local word reorderings are handled within phrases
  \item the more training data, the longer phrases can be learned reliably
  \end{itemize}
  \end{itemize}
\end{itemize}
\vfill

%%%%%%%%%%%%%%%%%%%%%%%%%%%%%%%%%%%%%%%%%%%%%%%%%%%%%%%%%%%%%%%%%%%%%%%%%%%%

\slide{Different Paradigms (2)}
\vfill
\begin{itemize}
\item {\bf Hierarchical translation}
  \begin{itemize}
  \item Allows for \emph{gaps} in the phrases
  \item Formalization as a synchronous context-free grammar (SCFG)
  \item Parsing-based decoding (\cykplus)
  \end{itemize}
\item {\bf Syntax-based translation}
  \begin{itemize}
  \item Comparable to hierarchical, but with \emph{linguistic well-formedness constraints}
  \item Source syntax, target syntax, or both
  \item SCFG or synchronous tree substitution grammar (STSG)
  \end{itemize}
\end{itemize}
\vfill

%%%%%%%%%%%%%%%%%%%%%%%%%%%%%%%%%%%%%%%%%%%%%%%%%%%%%%%%%%%%%%%%%%%%%%%%%%%%

\slide{Phrase-based Approach}
\begin{center} \vspace{15mm}
\includegraphics[scale=1.4]{phrase-model-alignment.pdf}
\end{center}\vspace{5mm}
\begin{itemize} \itemsep -1mm
\item Foreign input is segmented in phrases 
\item Each phrase is translated into English
\item Phrases are reordered
\end{itemize}

%%%%%%%%%%%%%%%%%%%%%%%%%%%%%%%%%%%%%%%%%%%%%%%%%%%%%%%%%%%%%%%%%%%%%%%%%%%%

% \slide{Data-driven MT}
% \vfill
% 
% \begin{center}\begin{tikzpicture}
% \node[processing, text width=12em, align=center](ptraining){Training};
% \node[input, text width=12em, align=center, above=1.5cm of ptraining](train){Parallel Training Data};
% \node[model, text width=12em, align=center, below=1cm of ptraining](tm){Translation Model};
% \draw[arr] (train) -- (ptraining);
% \draw[arr] (ptraining) -- (tm);
% 
% \node[input, text width=12em, align=center, right=2cm of train](monotrain){Monolingual Training Data};
% \node[model, text width=12em, align=center, right=2cm of tm] (lm) {Language Model};
% \node[processing, text width=12em, align=center, right=2cm of ptraining] (lmtrain) {LM Estimation};
% \draw[arr] (monotrain) -- (lmtrain);
% \draw[arr] (lmtrain) -- (lm);
% 
% \node[processing, text width=12em, align=center, below right=2.5cm  and 1cm of tm.center] (tuning) {Tuning};
% \node[input, text width=8em, align=center, right=2cm of tuning](dev){Development Set};
% \node[model, text width=12em, align=center, below=1.5cm of tuning] (system) {SMT System};
% \draw[arr, shorten >=4mm ] (lm) -- (tuning);
% \draw[arr, shorten >=4mm] (tm) -- (tuning);
% \draw[arr] (dev) -- (tuning);
% \draw[arr] (tuning) -- (system);
% 
% \end{tikzpicture}\end{center}
% 
% \vfill

%%%%%%%%%%%%%%%%%%%%%%%%%%%%%%%%%%%%%%%%%%%%%%%%%%%%%%%%%%%%%%%%%%%%%%%%%%%%

\slide{SMT Pipeline (1)}
\vfill
\begin{itemize}
  \item {\bf Data collection}
  \begin{itemize}
  \item Parallel corpora for training translation model (TM)
  \item Target-side monolingual corpora for training language model (LM)
  \item Separate development and test sets for tuning
  \end{itemize}
\item {\bf Data preparation, preprocessing}
  \begin{itemize}
  \item Cleaning: unrealiable parts should be removed
  \item Sentence segmentation, sentence alignment of parallel corpora
  \item Preprocessing steps (word tokenization, segmentation, casing, \ldots)
  \end{itemize}
\end{itemize}
\vfill

%%%%%%%%%%%%%%%%%%%%%%%%%%%%%%%%%%%%%%%%%%%%%%%%%%%%%%%%%%%%%%%%%%%%%%%%%%%%

\slide{SMT Pipeline (2)}
\vspace*{-1ex}
\begin{itemize}
\item {\bf Training}
  \begin{itemize}
  \item Word alignment
  \item Phrase extraction
  \item Language model estimation
  \item Lexicalized reordering model
  \item Single-word based lexicon model
  \item \ldots
  \end{itemize}
\item {\bf Tuning} 
  \begin{itemize}
  \item The decoder combines a set of features in a log-linear framework
  \item Each feature has a weight which determines its contribution to the overall model score
  \item We can optimize the values of the feature weights such that translation quality is maximized on a development set 
  \end{itemize}
\end{itemize}
%\vfill

%%%%%%%%%%%%%%%%%%%%%%%%%%%%%%%%%%%%%%%%%%%%%%%%%%%%%%%%%%%%%%%%%%%%%%%%%%%%

\slide{SMT Pipeline (3)}
\vspace*{-1ex}
\begin{itemize}
\item {\bf Decoding}
  \begin{itemize}
  \item Translate unseen source text
  \item Decoding $=$ searching the space of possible translations for the best hypothesis according to the model score
  \item Dynamic programming
  \item Pruning, recombination, and reordering constraints to keep the task feasible
  \end{itemize}
\vspace*{-0.5ex}
\item {\bf Postprocessing}
  \begin{itemize}
  \item Detokenization, recasing, \ldots
  \end{itemize}
\vspace*{-0.5ex}
\item {\bf Evaluation}
  \begin{itemize}
  \item Automatic metrics to assess translation quality by comparing the hypothesis with a reference translation: {\sc bleu}, {\sc ter}, {\sc meteor}, \ldots
  \item Human judgement
  \end{itemize}
\end{itemize}
\vspace*{-2ex}
%\vfill

%%%%%%%%%%%%%%%%%%%%%%%%%%%%%%%%%%%%%%%%%%%%%%%%%%%%%%%%%%%%%%%%%%%%%%

\slide{Data Collection}
%\vfill
\vspace*{-1ex}
\begin{center}
\includegraphics[height=\textheight]{nc-english.png}
\end{center}
%\vfill
%\vspace*{-2ex}

%%%%%%%%%%%%%%%%%%%%%%%%%%%%%%%%%%%%%%%%%%%%%%%%%%%%%%%%%%%%%%%%%%%%%%

\slide{Data Collection}
%\vfill
\vspace*{-1ex}
\begin{center}
\includegraphics[height=\textheight]{nc-german.png}
\end{center}
%\vfill
%\vspace*{-2ex}

%%%%%%%%%%%%%%%%%%%%%%%%%%%%%%%%%%%%%%%%%%%%%%%%%%%%%%%%%%%%%%%%%%%%%%

\slide{Sentence Segmentation}

\vfill
\small
\begin{framed}
Roxy Ann Peak is a 3,576-foot-tall (1,090 m) mountain in the Western Cascade Range in the U.S. state of Oregon. Composed of several geologic layers, the majority of the peak is of volcanic origin and dates to the early Oligocene. It is primarily covered by oak savanna and open grassland on its lower slopes, and mixed coniferous forest on its upper slopes and summit. Despite the peak's relatively small topographic prominence of 753 feet (230 m), it rises 2,200 feet (670 m) above Medford, and it is the city's most important viewshed, open space reserve, and recreational resource.
\end{framed}
\normalsize
\vfill

%%%%%%%%%%%%%%%%%%%%%%%%%%%%%%%%%%%%%%%%%%%%%%%%%%%%%%%%%%%%%%%%%%%%%%

\slide{Sentence Segmentation}

\vfill
%\vspace*{-1ex}
\small
\FrameSep5pt
\begin{framed}
Roxy Ann Peak is a 3,576-foot-tall (1,090 m) mountain in the Western Cascade Range in the U.S. state of Oregon.
\end{framed}
\begin{framed}
Composed of several geologic layers, the majority of the peak is of volcanic origin and dates to the early Oligocene.
\end{framed}
\begin{framed}
It is primarily covered by oak savanna and open grassland on its lower slopes, and mixed coniferous forest on its upper slopes and summit.
\end{framed}
\begin{framed}
Despite the peak's relatively small topographic prominence of 753 feet (230 m), it rises 2,200 feet (670 m) above Medford, and it is the city's most important viewshed, open space reserve, and recreational resource.
\end{framed}
\normalsize
\vfill

%%%%%%%%%%%%%%%%%%%%%%%%%%%%%%%%%%%%%%%%%%%%%%%%%%%%%%%%%%%%%%%%%%%%%%

\slide{Parallel Sentence Alignment}

\vfill
\small
\begin{addmargin}[-2em]{-1em}
\tikzstyle{corpus} =  [draw=black, fill = yellow!30, rounded corners, font=\small]
\begin{flushleft}
\begin{tikzpicture}

\node[corpus] (fr) {%
  \begin{minipage}{0.58\textwidth}
%  \scriptsize
Je vous invite \`a vous lever pour cette minute de silence.\\
(Le Parlement, debout, observe une minute de silence)\\
Madame la Pr\'esidente, c'est une motion de proc\'edure.
  \end{minipage}
};

\node[corpus] (en) [right of = fr, node distance = 0.57\textwidth] {%
  \begin{minipage}{0.5\textwidth}
%  \scriptsize
Please rise, then, for this minute's silence.\\
(The House rose and observed a minute's silence)\\
Madam President, on a point of order.
  \end{minipage}
};
\end{tikzpicture}
\end{flushleft}

\vfill

\tikzstyle{corpus} =  [draw=black, fill = blue!30, rounded corners, font=\small]
\begin{flushleft}
\begin{tikzpicture}

\node[corpus] (fr) {%
  \begin{minipage}{0.58\textwidth}
%  \scriptsize
  Astronomes Introduction Vid\'eo d'introduction\\ 
  Qu'est-ce que l'astronomie?\\
  Souvent consid\'er\'e comme la plus ancienne des sciences, elle d\'ecoule de notre \'etonnement et de nos questionnements envers le ciel L'astronomie est la science qui \'etudie l'Univers au-del\`a de l'atmosph\`ere terrestre.
  \end{minipage}
};
\node[corpus] (en) [right of = fr, node distance = 0.57\textwidth] {%
  \begin{minipage}{0.5\textwidth}
%  \scriptsize
  Astronomers Introduction Introduction video\\
  What is Astronomy?\\
  Often considered the oldest science, it was born of our amazement at the sky and our need to question Astronomy is the science of space beyond Earth's atmosphere.
  \end{minipage}
};
\end{tikzpicture}
\end{flushleft}
\end{addmargin}
\normalsize
\vfill
\begin{itemize}
\item Gale and Church algorithm or similar methods
\item E.g.\ Microsoft Bilingual Sentence Aligner
\end{itemize}
\vfill

%%%%%%%%%%%%%%%%%%%%%%%%%%%%%%%%%%%%%%%%%%%%%%%%%%%%%%%%%%%%%%%%%%%%%%%%%%%%

\slide{Word Alignment}

\vspace*{-1ex}
\begin{itemize}
\item After further preprocessing (tokenization, word segmentation), \\train word alignment
\vspace*{-2.5ex}
\item Typically IBM/HMM sequence of models with subsequent symmetrization
%\vspace*{-1ex}
%\item Or alternatives like discriminative word aligners
\end{itemize}
\vspace*{-3.7ex}

\begin{center}
\newlength{\alignsize}
\setlength{\alignsize}{1cm}
\begin{tikzpicture}
\tikzstyle{aligned}=[draw=black, anchor = south west,  rectangle, minimum height=\alignsize, minimum width=\alignsize, fill=black]
\tikzstyle{english}=[minimum height=\alignsize, anchor = south west, node distance=\alignsize, text width =3cm, inner sep = 0pt, outer sep = 0pt]
\tikzstyle{foreign}=[minimum height=\alignsize, minimum width=3cm, text width = 3cm, anchor = north east, rotate = 90, node distance=\alignsize]
\draw  (0,0) grid +(9,7);
\node [aligned] at (0,6) {};
\node [aligned] at (1,4) {};
\node [aligned] at (1,5) {};
\node [aligned] at (2,3) {};
\node [aligned] at (3,3) {};
\node [aligned] at (4,3) {};
\node [aligned] at (5,2) {};
\node [aligned] at (6,2) {};
\node [aligned] at (7,0) {};
\node [aligned] at (8,1) {};
\node[english] (e1) at (-2,6) {Mary};
\node[english] (e2) [below of = e1] {did};
\node[english] (e3) [below of = e2]  {not};
\node[english] (e4) [below of = e3]  {slap};
\node[english] (e5) [below of = e4]  {the};
\node[english] (e6) [below of = e5]  {green};
\node[english] (e7) [below of = e6]  {witch};
\node[foreign] (f1) at (0,10.5) {Mar\'ia};
\node[foreign] (f2) [below of = f1] {no};
\node[foreign] (f3) [below of = f2] {daba};
\node[foreign] (f4) [below of = f3] {una};
\node[foreign] (f5) [below of = f4] {bofetada};
\node[foreign] (f6) [below of = f5] {a};
\node[foreign] (f7) [below of = f6] {la};
\node[foreign] (f8) [below of = f7] {bruja};
\node[foreign] (f9) [below of = f8] {verde};
\end{tikzpicture}
%\includegraphics[scale=1.2]{michael-alignment.pdf}
\end{center}
\vspace*{-2ex}

%%%%%%%%%%%%%%%%%%%%%%%%%%%%%%%%%%%%%%%%%%%%%%%%%%%%%%%%%%%%%%%%%%%%%%%%%%%%

\slide{Phrase Extraction}
\begin{center}
\includegraphics[width=0.39\textwidth]{waiph1.pdf}
\end{center}
\vspace{-1mm}
{\bf \footnotesize \textcolor{blue}{ 
(Maria, Mary),
(no, did not),
(slap, daba una bofetada),
(a la, the),
(bruja, witch),
(verde, green) \\ \vspace{-1mm}
} }

%%%%%%%%%%%%%%%%%%%%%%%%%%%%%%%%%%%%%%%%%%%%%%%%%%%%%%%%%%%%%%%%%%%%%%%%%%%%

\slide{Phrase Extraction}
\begin{center}
\includegraphics[width=0.39\textwidth]{waiph2.pdf}
\end{center}
\vspace{-1mm}
{\bf \footnotesize \textcolor{blue}{
(Maria, Mary),
(no, did not),
(slap, daba una bofetada),
(a la, the),
(bruja, witch),
(verde, green),\\ \vspace{-1mm}
} \textcolor{green}{
(Maria no, Mary did not),
(no daba una bofetada, did not slap), 
(daba una bofetada a la, slap the),\\ \vspace{-1mm}
(bruja verde, green witch) \\ \vspace{-1mm} 
} }

%%%%%%%%%%%%%%%%%%%%%%%%%%%%%%%%%%%%%%%%%%%%%%%%%%%%%%%%%%%%%%%%%%%%%%%%%%%%

\slide{Phrase Extraction}
\begin{center}
\includegraphics[width=0.39\textwidth]{waiph3.pdf}
\end{center}
\vspace{-1mm}
{\bf \footnotesize \textcolor{blue}{
(Maria, Mary),
(no, did not),
(slap, daba una bofetada),
(a la, the),
(bruja, witch),
(verde, green), \\ \vspace{-1mm}
} \textcolor{green}{
(Maria no, Mary did not),  
(no daba una bofetada, did not slap), 
(daba una bofetada a la, slap the), \\ \vspace{-1mm}
(bruja verde, green witch),
} \textcolor{red}{
(Maria no daba una bofetada, Mary did not slap), \\ \vspace{-1mm}
(no daba una bofetada a la, did not slap the),
(a la bruja verde, the green witch) \\ \vspace{-1mm}
} }

%%%%%%%%%%%%%%%%%%%%%%%%%%%%%%%%%%%%%%%%%%%%%%%%%%%%%%%%%%%%%%%%%%%%%%%%%%%%

\slide{Phrase Extraction}
\begin{center}
\includegraphics[width=0.39\textwidth]{waiph4.pdf}
\end{center}
\vspace{-1mm}
{\bf \footnotesize \textcolor{blue}{
(Maria, Mary),
(no, did not),
(slap, daba una bofetada),
(a la, the),
(bruja, witch), 
(verde, green),
} \textcolor{green}{ \\ \vspace{-1mm}
(Maria no, Mary did not), 
(no daba una bofetada, did not slap), 
(daba una bofetada a la, slap the), \\ \vspace{-1mm}
(bruja verde, green witch), 
} \textcolor{red}{
(Maria no daba una bofetada, Mary did not slap), \\ \vspace{-1mm}
(no daba una bofetada a la, did not slap the),
(a la bruja verde, the green witch), \\ \vspace{-1mm}
} \textcolor{magenta}{
(Maria no daba una bofetada a la, Mary did not slap the), \\ \vspace{-1mm}
(daba una bofetada a la bruja verde, slap the green witch) \\ \vspace{-1mm}
} }

%%%%%%%%%%%%%%%%%%%%%%%%%%%%%%%%%%%%%%%%%%%%%%%%%%%%%%%%%%%%%%%%%%%%%%%%%%%%

\slide{Phrase Extraction}
\begin{center}
\includegraphics[width=0.39\textwidth]{waiph5.pdf}
\end{center}
\vspace{-1mm}
{\bf \footnotesize \textcolor{blue}{
(Maria, Mary),
(no, did not),
(slap, daba una bofetada),
(a la, the),
(bruja, witch), 
(verde, green),
} \textcolor{green}{ \\ \vspace{-1mm}
(Maria no, Mary did not), 
(no daba una bofetada, did not slap), 
(daba una bofetada a la, slap the), \\ \vspace{-1mm}
(bruja verde, green witch),
} \textcolor{red}{
(Maria no daba una bofetada, Mary did not slap), \\ \vspace{-1mm}
(no daba una bofetada a la, did not slap the),
(a la bruja verde, the green witch), \\ \vspace{-1mm}
} \textcolor{magenta}{
(Maria no daba una bofetada a la, Mary did not slap the), 
(daba una bofetada a la bruja verde, \\ \vspace{-1mm} slap the green witch), 
} \textcolor{orange}{
(no daba una bofetada a la bruja verde, did not slap the green
witch), \\ \vspace{-1mm}
} \textcolor{cyan}{
 (Maria no daba una bofetada a la bruja verde, Mary did not
slap the green witch)
} }

%%%%%%%%%%%%%%%%%%%%%%%%%%%%%%%%%%%%%%%%%%%%%%%%%%%%%%%%%%%%%%%%%%%%%%%%%%%%

\slide{Phrase Translation Probabilities}
\vfill
\begin{itemize}
\item {\bf Phrase table}: contains all phrases $(\overline{e} , \overline{f})$ extracted from the data
\item We need to assign {\bf probabilities} to extracted phrases
\item Score by relative frequency:
%\vfill
\maths{\begin{equation*}
\begin{split} 
  \phi(\bar{e}|\bar{f}) = \frac{\mbox{count}(\bar{e},\bar{f})}{\sum_{\bar{e}'}\mbox{count}(\bar{e}',\bar{f})}\\ \\
\phi(\bar{f}|\bar{e}) = \frac{\mbox{count}(\bar{e},\bar{f})}{\sum_{\bar{f}'}\mbox{count}(\bar{e},\bar{f}')}
\end{split} 
\end{equation*}}
\end{itemize}
\vfill

%%%%%%%%%%%%%%%%%%%%%%%%%%%%%%%%%%%%%%%%%%%%%%%%%%%%%%%%%%%%%%%%%%%%%%%%%%%%

\slide{Real Example}
\vfill
\begin{itemize}
\item Phrase translations for \example{den Vorschlag} from German into English: % learned from the Europarl corpus:
\end{itemize}
\vfill
\begin{center}
\begin{tabular}{|l|r||l|r|} \hline
{\bf English} & \maths{$\phi(\bar{e}|\bar{f})$} &
{\bf English} & \maths{$\phi(\bar{e}|\bar{f})$} \\ \hline \hline
\example{the proposal} & 0.6227 &
\example{the suggestions} & 0.0114 \\ \hline
\example{'s proposal} & 0.1068 &
\example{the proposed} & 0.0114 \\ \hline
\example{a proposal} & 0.0341 &
\example{the motion} & 0.0091 \\ \hline
\example{the idea} & 0.0250 &
\example{the idea of} & 0.0091 \\ \hline
\example{this proposal} & 0.0227 &
\example{the proposal ,} & 0.0068 \\ \hline
\example{proposal} & 0.0205 &
\example{its proposal} & 0.0068 \\ \hline
\example{of the proposal} & 0.0159 &
\example{it} & 0.0068 \\ \hline
\example{the proposals} & 0.0159 &
... & ... \\ \hline
\end{tabular}
\end{center}
\vfill

%%%%%%%%%%%%%%%%%%%%%%%%%%%%%%%%%%%%%%%%%%%%%%%%%%%%%%%%%%%%%%%%%%%%%%%%%%%%

\slide{More Features}
\vspace*{-1ex}
\begin{itemize} \itemsep -0.2ex
\item The decoder will employ the phrase translation probabilities along with other features to assign an overall score to each possible hypothesis
\item {\bf Standard features}:
\vspace*{-1ex}
\begin{itemize}
  \item Phrase translation log-probabilities
  \item Lexical translation log-probabilities (from single-word based models)
  \item Language model log-probability (from a target-side $n$-gram LM)
  \item Phrase count
  \item Word count
  \item Distortion cost based on jump distances
  \item Lexicalized reordering score
\end{itemize}
\item {\bf Feature types}:
\vspace*{-1ex}
\begin{itemize}
  \item phrase-local vs.\ context-aware
  \item stateless vs.\ stateful
  \item dense vs.\ sparse
\end{itemize}
\end{itemize}
\vspace*{-2ex}

%%%%%%%%%%%%%%%%%%%%%%%%%%%%%%%%%%%%%%%%%%%%%%%%%%%%%%%%%%%%%%%%%%%%%%%%%%%%

\slide{Log-linear Model}
\vfill
\begin{itemize} %\itemsep -2em
\item Decoding: Given the model, find the best translation
\vspace*{-0.7ex}
\maths{
\begin{equation*}
  {\bf e}_{\mbox{\small best}} =  \; \mbox{argmax}_{\bf e} \; p({\bf e}|{\bf f})
\vspace*{-0.5ex}
\end{equation*}
}
\item Our model is a {\bf weighted combination} of many components
\vspace*{-0.7ex}
\maths{
\begin{equation*}
p({\bf e}|{\bf f}) \propto \exp \sum_{k=1}^m \; \lambda_k\cdot h_k({\bf e}, {\bf f}) 
\vspace*{-0.5ex}
\end{equation*}
}
where $h_k({\bf e}, {\bf f})$ are {\bf feature functions} and $\lambda_k$ are {\bf feature weights}
\end{itemize}
\vfill

%%%%%%%%%%%%%%%%%%%%%%%%%%%%%%%%%%%%%%%%%%%%%%%%%%%%%%%%%%%%%%%%%%%%%%%%%%%%

\slide{Tuning}
\vspace*{2ex}
\begin{center}
\includegraphics[width=0.8\textwidth]{discriminative.pdf}
\end{center}
\begin{itemize} %\itemsep -2em
\item MERT, MIRA, PRO, \ldots
\end{itemize}
\vspace*{-2ex}

%%%%%%%%%%%%%%%%%%%%%%%%%%%%%%%%%%%%%%%%%%%%%%%%%%%%%%%%%%%%%%%%%%%%%%%%%%%%%

\slide{Phrase-based Decoding}
\vspace*{-1ex}
%\begin{center} 
\includegraphics[scale=1.5]{translation-options.pdf}
%\end{center}\vspace{-10mm}
\begin{itemize}
\vspace*{-4ex}
\item Many translation options to choose from
\end{itemize}
\vspace*{-2ex}

%%%%%%%%%%%%%%%%%%%%%%%%%%%%%%%%%%%%%%%%%%%%%%%%%%%%%%%%%%%%%%%%%%%%%%%%%%%%%

\slide{Phrase-based Decoding}
\vspace*{-1ex}
%\begin{center} 
\includegraphics[scale=1.5]{translation-options-correct.pdf}
%\end{center}  \vspace{-12mm}
\vspace*{-4ex}
\begin{itemize}
\item The machine translation decoder does not know the right answer\vspace{-4mm}
\begin{itemize}
\item picking the right translation options
\item arranging them in the right order \vspace{-6mm}
\end{itemize}
\item[$\rightarrow$] Search problem solved by beam search
\end{itemize}
\vspace*{-2ex}

%%%%%%%%%%%%%%%%%%%%%%%%%%%%%%%%%%%%%%%%%%%%%%%%%%%%%%%%%%%%%%%%%%%%%%%%%%%%%

\slide{\hspace*{-2em}Decoding: Precompute Translation Options}
\begin{center}
\includegraphics[scale=1.3]{decoding-step1.pdf}\\[69mm]
consult phrase translation table for all input phrases
\end{center}

%%%%%%%%%%%%%%%%%%%%%%%%%%%%%%%%%%%%%%%%%%%%%%%%%%%%%%%%%%%%%%%%%%%%%%%%%%%%

\slide{Decoding: Start with Initial Hypothesis}
\begin{center} 
\includegraphics[scale=1.3]{decoding-step2.pdf}\\[22mm]
initial hypothesis: no input words covered, no output produced
\end{center}

%%%%%%%%%%%%%%%%%%%%%%%%%%%%%%%%%%%%%%%%%%%%%%%%%%%%%%%%%%%%%%%%%%%%%%%%%%%%

\slide{Decoding: Hypothesis Expansion}
\begin{center}
\includegraphics[scale=1.3]{decoding-step3.pdf}\\[22mm]
pick any translation option, create new hypothesis
\end{center} 

%%%%%%%%%%%%%%%%%%%%%%%%%%%%%%%%%%%%%%%%%%%%%%%%%%%%%%%%%%%%%%%%%%%%%%%%%%%%

\slide{Decoding: Hypothesis Expansion}
\begin{center} 
\includegraphics[scale=1.3]{decoding-step4.pdf}\\[5mm]
create hypotheses for all other translation options
\end{center}

%%%%%%%%%%%%%%%%%%%%%%%%%%%%%%%%%%%%%%%%%%%%%%%%%%%%%%%%%%%%%%%%%%%%%%%%%%%%

\slide{Decoding: Hypothesis Expansion}
\begin{center} 
\includegraphics[scale=1.3]{decoding-step5.pdf}\\
also create hypotheses from created partial hypotheses
\end{center}

%%%%%%%%%%%%%%%%%%%%%%%%%%%%%%%%%%%%%%%%%%%%%%%%%%%%%%%%%%%%%%%%%%%%%%%%%%%%

\slide{Decoding: Find Best Path}
\begin{center}
\includegraphics[scale=1.3]{decoding-step6.pdf}\\[1mm]
backtrack from highest scoring complete hypothesis
\end{center}

%%%%%%%%%%%%%%%%%%%%%%%%%%%%%%%%%%%%%%%%%%%%%%%%%%%%%%%%%%%%%%%%%%%%%%%%%%%%

\slide{Pruning}
\vfill
\begin{itemize}
\item Heuristically discard weak hypotheses early: {\bf beam search}
\item Organize hypotheses in {\bf stacks} (actually priority queues), e.g.\ by
{\begin{itemize} \vspace{-4mm}
\item same source words covered
\item same number of source words covered 
%\item \important{same number} of English words produced
\end{itemize} }
\item Compare hypotheses in stacks, discard bad ones
{\begin{itemize} \vspace{-4mm}
\item {\bf histogram pruning}: keep top $k$ hypotheses in each stack (e.g., $k$=100)
\item {\bf threshold pruning}: keep hypotheses that are at most $\alpha$
  times the cost of best hypothesis in stack (e.g., $\alpha$ = 0.001)
\end{itemize} }
\end{itemize}
\vfill

%%%%%%%%%%%%%%%%%%%%%%%%%%%%%%%%%%%%%%%%%%%%%%%%%%%%%%%%%%%%%%%%%%%%%%%%%%%%

\slide{Hypothesis Stacks}
\vfill
\begin{center}
\includegraphics[width=15cm]{hypothesis-stacks.pdf}
\end{center}
\begin{itemize}
\item Organization of hypotheses into stacks
{\begin{itemize}
  \item here: based on {\bf number of source words} translated
\item during translation all hypotheses from one stack are expanded
\item expanded hypotheses are placed into next stacks
\end{itemize} }
\end{itemize}
\vfill

%%%%%%%%%%%%%%%%%%%%%%%%%%%%%%%%%%%%%%%%%%%%%%%%%%%%%%%%%%%%%%%%%%%%%%%%%%%%

\slide{Evaluation}
\vfill
\begin{itemize}
\item {\bf Automatic evaluation metrics}
\begin{itemize}
\item Measure similarity between machine translation output \\and human-generated reference translation
\item Good metrics correlate well with human judgement
\end{itemize}
\item Decode a held-out test set and -- after postprocessing (detokenization, truecasing) -- score with an automatic metric to {\bf determine translation quality}
\item {\bf \sc bleu}
\begin{itemize}
\item Most commonly used metric
\item $n$-gram precision ($n = 1 \ldots 4$), brevity penalty
\end{itemize}
%\item {\bf \sc ter}
%\item {\bf \sc meteor}
\end{itemize}
\vfill

%%%%%%%%%%%%%%%%%%%%%%%%%%%%%%%%%%%%%%%%%%%%%%%%%%%%%%%%%%%%%%%%%%%%%%%%%%%%

%\slide{Computational Complexity}
%\begin{itemize}\vspace{35mm} \itemsep 5mm
%\item The suggested process creates exponential number of hypothesis 
%\item Reduction of search space: pruning
%\item[$\rightarrow$] Decoder may not find the model-best translation
%\end{itemize}

%%%%%%%%%%%%%%%%%%%%%%%%%%%%%%%%%%%%%%%%%%%%%%%%%%%%%%%%%%%%%%%%%%%%%%%%%%%%

% \slide{Factored Translation}
% \begin{itemize}
% \item Factored represention of words 
% \begin{center} \vspace{-8mm}
% \includegraphics[width=155mm]{factors.pdf}
% \end{center} \vspace{-22mm}
% \item Goals  \vspace{-3mm}
% \begin{itemize}
% \item generalization, e.g. by translating lemmas, not surface forms
% \item richer model, e.g. using syntax for reordering, language modeling
% \end{itemize}
% \end{itemize}

%%%%%%%%%%%%%%%%%%%%%%%%%%%%%%%%%%%%%%%%%%%%%%%%%%%%%%%%%%%%%%%%%%%%%%%%%%%%

% \slide{Factored Model}
% \begin{center}
% Example:\\
% \includegraphics[scale=1.6]{factored-morphgen.pdf}\\
% Decomposing the translation step\\
% Translating lemma and morphological information more robust
% \end{center}

%%%%%%%%%%%%%%%%%%%%%%%%%%%%%%%%%%%%%%%%%%%%%%%%%%%%%%%%%%%%%%%%%%%%%%%%%%%%

\slide{Hierarchical Translation}
%\begin{center}
\hspace*{3em} 
\includegraphics[scale=1.4]{hierarchical-phrase-extraction1.pdf}
%\end{center}

%%%%%%%%%%%%%%%%%%%%%%%%%%%%%%%%%%%%%%%%%%%%%%%%%%%%%%%%%%%%%%%%%%%%%%%%%%%%

\slide{Hierarchical Translation}
%\begin{center}
\hspace*{3em} 
\includegraphics[scale=1.4]{hierarchical-phrase-extraction2.pdf}
%\end{center}

%%%%%%%%%%%%%%%%%%%%%%%%%%%%%%%%%%%%%%%%%%%%%%%%%%%%%%%%%%%%%%%%%%%%%%%%%%%%

\slide{Hierarchical Translation}
%\begin{center}
\hspace*{3em} 
\includegraphics[scale=1.4]{hierarchical-phrase-extraction3.pdf}
%\end{center}

%%%%%%%%%%%%%%%%%%%%%%%%%%%%%%%%%%%%%%%%%%%%%%%%%%%%%%%%%%%%%%%%%%%%%%%%%%%%

\slide{Hierarchical Translation}
%\begin{center}
\hspace*{3em} 
\includegraphics[scale=1.4]{hierarchical-phrase-extraction4.pdf}
%\end{center}

%%%%%%%%%%%%%%%%%%%%%%%%%%%%%%%%%%%%%%%%%%%%%%%%%%%%%%%%%%%%%%%%%%%%%%%%%%%%

\slide{Syntax-based Translation}
%\vspace*{-1em} 
%\begin{center}
\begin{tabular}{rl}
\includegraphics[scale=1.17]{hierarchical-phrase-extraction-syntax1.pdf}
& 
\parbox{85mm}{\footnotesize \phantom{.}\vspace{-6cm}

\example{\Tree [.{\sc pro} Ihnen ] }
\hspace{2mm} {\Large $=$} \hspace{2mm}
\example{\Tree [.{\sc pp} [.{\sc to} to ] [.{\sc prp} you ] ] \vspace{-1cm}\phantom{.}}
}
\end{tabular}
%\end{center}
\vspace*{-2em} 

%%%%%%%%%%%%%%%%%%%%%%%%%%%%%%%%%%%%%%%%%%%%%%%%%%%%%%%%%%%%%%%%%%%%%%%%%%%%

% \slide{Constraints on Syntactic Rules}
% \vspace{15mm}
% \begin{itemize}
% \item Same word alignment constraints as hierarchical models
% \item Hierarchical: rule can cover any span\\
% $\Leftrightarrow$ syntactic rules must cover constituents in the tree
% \item Hierarchical: gaps may cover any span\\
% $\Leftrightarrow$ gaps must cover constituents in the tree
% \vspace{10mm}
% \item Much less rules are extracted (all things being equal)
% \end{itemize}

%%%%%%%%%%%%%%%%%%%%%%%%%%%%%%%%%%%%%%%%%%%%%%%%%%%%%%%%%%%%%%%%%%%%%%%%%%%%

\slide{Syntax-based Translation: Variants}
\vspace{5mm}
{\small
\begin{center}
\begin{tabular}{cc}
\textcolor{black}{\bf String-to-String} & \textcolor{black}{\bf String-to-Tree} \\
\example{John misses Mary}
& \example{John misses Mary}\\
$\Rightarrow$  Marie manque \`a Jean
& $\Rightarrow$ \hspace{-2cm} \Tree [.S [.NP \example{Marie} ] [.VP [.V \example{manque} ] [.PP [.P \example{\`a} ] [.NP \example{Jean} ]  ]  ]  ] \\[5mm]

\textcolor{black}{\bf Tree-to-String} & \textcolor{black}{\bf Tree-to-Tree} \\
\Tree [.S [.NP \example{John} ] [.VP [.V \example{misses} ] [.NP \example{Mary} ]  ]  ] 
&  \Tree [.S [.NP \example{John} ] [.VP [.V \example{misses} ] [.NP \example{Mary} ]  ]  ] \hspace{-1cm} $\Rightarrow$ \hspace{-2cm}\Tree [.S [.NP \example{Marie} ] [.VP [.V \example{manque} ] [.PP [.P \example{\`a} ] [.NP \example{Jean} ]  ]  ]  ] \\[-1cm]
$\Rightarrow$ \example{Marie manque \`a Jean}  	
& 
\end{tabular}
\end{center}
}

%%%%%%%%%%%%%%%%%%%%%%%%%%%%%%%%%%%%%%%%%%%%%%%%%%%%%%%%%%%%%%%%%%%%%%%%%%%%

\slide{Chart Parsing}
\begin{addmargin}[-1em]{-1em}
\begin{tabular}{lr}
  \vspace{0pt} \includegraphics[scale=1]{chart-parsing-stacks.pdf} &
  \vspace{0pt} \raisebox{12ex}{\includegraphics[scale=0.7]{chart-stacks-color.pdf}}
\end{tabular}
\end{addmargin}
\vspace*{-1ex}
\begin{itemize} \itemsep -0.2ex
\item {\bf Bottom-up chart parsing} with \cykplus
\item Chart cells cover continuous spans of the source sentence
\item Filling chart cells with hypotheses: {\bf cube pruning}
\end{itemize}
\vspace*{-2ex}

%%%%%%%%%%%%%%%%%%%%%%%%%%%%%%%%%%%%%%%%%%%%%%%%%%%%%%%%%%%%%%%%%%%%%%%%%%%%

\slide{Syntax-based Decoding}
\vspace{-25mm}
\begin{center}
\includegraphics[scale=1.15]{chart-parsing0.pdf}
\end{center}
\vspace*{-2ex}

%%%%%%%%%%%%%%%%%%%%%%%%%%%%%%%%%%%%%%%%%%%%%%%%%%%%%%%%%%%%%%%%%%%%%%%%%%%%

\slide{Syntax-based Decoding}
\vspace{-25mm}
\begin{center}
\includegraphics[scale=1.15]{chart-parsing1.pdf}
\end{center}
\vspace*{-2ex}

%%%%%%%%%%%%%%%%%%%%%%%%%%%%%%%%%%%%%%%%%%%%%%%%%%%%%%%%%%%%%%%%%%%%%%%%%%%%

\slide{Syntax-based Decoding}
\vspace{-25mm}
\begin{center}
\includegraphics[scale=1.15]{chart-parsing2.pdf}
\end{center}
\vspace*{-2ex}

%%%%%%%%%%%%%%%%%%%%%%%%%%%%%%%%%%%%%%%%%%%%%%%%%%%%%%%%%%%%%%%%%%%%%%%%%%%%

\slide{Syntax-based Decoding}
\vspace{-25mm}
\begin{center}
\includegraphics[scale=1.15]{chart-parsing3.pdf}
\end{center}
\vspace*{-2ex}

%%%%%%%%%%%%%%%%%%%%%%%%%%%%%%%%%%%%%%%%%%%%%%%%%%%%%%%%%%%%%%%%%%%%%%%%%%%%

\slide{Syntax-based Decoding}
\vspace{-25mm}
\begin{center}
\includegraphics[scale=1.15]{chart-parsing4.pdf}
\end{center}
\vspace*{-2ex}

%%%%%%%%%%%%%%%%%%%%%%%%%%%%%%%%%%%%%%%%%%%%%%%%%%%%%%%%%%%%%%%%%%%%%%%%%%%%

\slide{Syntax-based Decoding}
\vspace{-25mm}
\begin{center}
\includegraphics[scale=1.15]{chart-parsing5.pdf}
\end{center}
\vspace*{-2ex}

%%%%%%%%%%%%%%%%%%%%%%%%%%%%%%%%%%%%%%%%%%%%%%%%%%%%%%%%%%%%%%%%%%%%%%%%%%%%

\slide{Syntax-based Decoding}
\begin{center}
\vspace{-29mm}
\includegraphics[scale=1.15]{chart-parsing.pdf}
\vspace*{-2ex}
\end{center}

%%%%%%%%%%%%%%%%%%%%%%%%%%%%%%%%%%%%%%%%%%%%%%%%%%%%%%%%%%%%%%%%%%%%%%%%%%%%

% \slide{}
% \vspace{-2.5cm}
% \begin{center}
% \includegraphics[width=26cm]{ems-syntax-viz.png}
% \end{center}

%%%%%%%%%%%%%%%%%%%%%%%%%%%%%%%%%%%%%%%%%%%%%%%%%%%%%%%%%%%%%%%%%%%%%%%%%%%%

\slide{How do I get started?}

\vfill
\begin{itemize} %\itemsep -1mm
\item {\bf Set up Moses}
%   \begin{itemize}
%   \item Download source code for Moses, GIZA++, MGIZA
%   \item Compile, install
%   \item Or use precompiled Moses packages (Windows, Linux, Mac\,OS\,X)
%         \begin{itemize}
%         \item Installer for Windows with GUI
%         \item Debian Linux RPMs
%         \item Binaries for Cygwin/Linux/Mac\,OS\,X
%         \end{itemize}
%   \item More info: \url{http://www.statmt.org/moses/}
%   \end{itemize}
\item {\bf Collect your data}
  \begin{itemize}
%  \item Parallel data 
%        \begin{itemize}
%        \item Freely available data, e.g.\ Europarl, MultiUN, WIT3, OPUS, \ldots
        \item Available corpora like Europarl, MultiUN, WIT3, OPUS, CommonCrawl
        \item Data from open MT evaluation campaigns (WMT, IWSLT, \ldots)
        \item TAUS, Linguistic Data Consortium (LDC), \ldots
%        \end{itemize}
%  \item Monolingual data
%        \begin{itemize}
%        \item CommonCrawl, WMT News Crawl corpus, \ldots
%        \end{itemize}
  \item If data is not prepared yet, carry out cleaning, sentence segmentation, and parallel sentence alignment
%  \item Translation memories
  \end{itemize}
\item {\bf Train system}
\end{itemize}
\vfill

%%%%%%%%%%%%%%%%%%%%%%%%%%%%%%%%%%%%%%%%%%%%%%%%%%%%%%%%%%%%%%%%%%%%%%%%%%%%

\slide{Open Source Components}
\vspace*{-1ex}
\begin{itemize}
\item {\bf Moses} %statistical machine translation system}
  \begin{itemize}
  \item \url{http://www.statmt.org/moses/}
  \item LGPL license
  \item Get stable release or snapshot from GitHub % \url{https://github.com/moses-smt/mosesdecoder}
  \item User manual % \url{http://www.statmt.org/moses/manual/manual.pdf}
        and support mailing list % \url{http://mailman.mit.edu/mailman/listinfo/moses-support}
  \end{itemize}
\vspace*{-0.5ex}
\item {\bf External tools used by Moses}
  \begin{itemize}
  \item Word alignment: {\sc giza++}, {\sc mgiza}, {\sc BerkeleyAligner}, {\sc FastAlign}
  \item Language model: {\sc kenlm}, {\sc irstlm}, {\sc srilm}
  \item Evaluation: {\sc bleu}, {\sc ter}, {\sc meteor}
  \end{itemize}
\vspace*{-0.5ex}
\item {\bf Other useful tools}
  \begin{itemize}
  \item Sentence aligner, syntactic parsers, part-of-speech taggers, \\morphological analyzers
  \end{itemize}
\vspace*{-2ex}
\end{itemize}

%%%%%%%%%%%%%%%%%%%%%%%%%%%%%%%%%%%%%%%%%%%%%%%%%%%%%%%%%%%%%%%%%%%%%%%%%%%%

\slide{Moses Pipeline}
%\vspace*{-1ex}
\vfill
\begin{itemize}
\item Plenty steps, each with lots of possible configuration parameters
\vfill
\hspace*{2em}
%\begin{center}
\begin{SaveVerbatim}{myverb}
tokenize < corpus.en > corpus.en.tok
lowercase < corpus.en.tok > corpus.en.lc
...
train-model.perl ...
mert.perl ...
moses ...
mteval-v13.pl ...
\end{SaveVerbatim}
\colorbox{gray}{\BUseVerbatim{myverb}}
%\end{center}
%\item Change a part of the process, execute everything again
\end{itemize}
\vfill

%%%%%%%%%%%%%%%%%%%%%%%%%%%%%%%%%%%%%%%%%%%%%%%%%%%%%%%%%%%%%%%%%%%%%%%%%%%%

% \slide{Phrase-based Model Training with Moses}
% \vspace{20mm}
% \begin{itemize}
% \item Command line
% \begin{center}
% \begin{SaveVerbatim}{myverb}
% train-model.perl ...
% \end{SaveVerbatim}
% \colorbox{gray}{\BUseVerbatim{myverb}}
% \end{center}
% 
% \item Example phrase from model
% \begin{center}
% \footnotesize
% \begin{SaveVerbatim}{myverb}
% B�ndnisse ||| alliances ||| 1 1 1 1 2.718 ||| ||| 1 1
% General Musharraf betrat am ||| general Musharraf appeared on ||| 1 1 1 1 2.718 ||| ||| 1 1
% 
% \end{SaveVerbatim}
% \colorbox{gray}{\BUseVerbatim{myverb}}
% \end{center}
% \end{itemize}
% 
% % phrase based
% %  /Users/hieuhoang/workspace/sourceforge/trunk/scripts/training/train-model.perl -first-step 5 -last-step 6 --corpus corpus.1.0-0 -f de -e en -alignment-file /Users/hieuhoang/workspace/data/syntax-de-en-nc/10/aligned.1 -alignment  grow-diag-final-and -lexical-file lex 

%%%%%%%%%%%%%%%%%%%%%%%%%%%%%%%%%%%%%%%%%%%%%%%%%%%%%%%%%%%%%%%%%%%%%%%%%%%%

% \slide{Phrase-based Decoding with Moses}
% 
% \begin{itemize}
% \item Command line   
% \vspace{-4mm}
% \begin{center}
% \begin{SaveVerbatim}{myverb}
% moses -f moses.ini -i in.txt > out.txt
% \end{SaveVerbatim}
% \colorbox{gray}{\BUseVerbatim{myverb}}
% \end{center}
% %\item Output
% %\vspace{-4mm}
% %\begin{center}
% %\begin{SaveVerbatim}{myverb}
% %there are various different opinions .
% %\end{SaveVerbatim}
% %\colorbox{gray}{\BUseVerbatim{myverb}}
% %\end{center}
% \item Advantages
% \vspace{-1mm}
% \begin{itemize}
% \item fast --- under half a second per sentence for fast configuration
% \item low memory requirements --- $\sim$200MB RAM for lowest configuration
% %	\begin{itemize}
% %	\item 200-300MB for lowest configuration
% %	\item suitable for netbooks and mobile devices
% %	\end{itemize}
% \item state-of-the-art translation quality on most tasks,\\ especially for related language pairs
% \item robust --- does not rely on any syntactic annotation
% \end{itemize}
% \item Disadvantages
% \vspace{-1mm}
% \begin{itemize}
% \item poor modeling of linguistic knowledge and of long-distance dependencies
% \end{itemize}
% \end{itemize}

%%%%%%%%%%%%%%%%%%%%%%%%%%%%%%%%%%%%%%%%%%%%%%%%%%%%%%%%%%%%%%%%%%%%%%%%%%%%

% \slide{Hierarchical Model Training with Moses}
% %\vspace{5mm}
% \begin{itemize}
% \item Hierarchical model:\\ formally syntax-based, without linguistic annotation (string-to-string)
% %\vspace{-5mm}
% \item Command line
% \begin{center}
% \begin{SaveVerbatim}{myverb} 
% train-model.perl -hierarchical ...
% \end{SaveVerbatim}
% \colorbox{gray}{\BUseVerbatim{myverb}}
% \end{center}
% %\vspace{-5mm}
% \item Example rule from model
% \begin{center}
% \vspace{-15mm}
% \littlecode{\tiny B\"undnisse [X][X] Kr\"aften [X] ||| alliances [X][X] forces [X] ||| 1 1 1 1 2.718 ||| 1-1 ||| 0.0526316 0.0526316}
% \end{center}
% \item Visualization of rule
% \vspace{-10mm}
% \begin{center}
% \tikzset{level distance=72pt}
% \Tree [.X [. B\"undnisse ]  [.X$_1$ ] [. Kr\"aften ] ] $\Rightarrow$ \Tree [.X [. alliances ]  [.X$_1$ ] [. forces ] ]
% \end{center}
% \end{itemize}

%%%%%%%%%%%%%%%%%%%%%%%%%%%%%%%%%%%%%%%%%%%%%%%%%%%%%%%%%%%%%%%%%%%%%%%%%%%%

% \slide{Hierarchical Decoding with Moses}
% 
% \begin{itemize}
% \item Command line   
% \vspace{-4mm}
% \begin{center}
% \begin{SaveVerbatim}{myverb}
% moses_chart -f moses.ini -i in.txt > out.txt
% \end{SaveVerbatim}
% \colorbox{gray}{\BUseVerbatim{myverb}}
% \end{center}
% 
% %  /Users/hieuhoang/workspace/sourceforge/trunk/scripts/training/train-model.perl -first-step 5 -last-step 6 --corpus corpus.1.0-0 -f de -e en -alignment-file /Users/hieuhoang/workspace/data/syntax-de-en-nc/10/aligned.1 -alignment  grow-diag-final-and -lexical-file lex  -hierarchical
% 
% \item Advantages
% \begin{itemize}
% \item able to model non-contiguities like \emph{ne \ldots pas $\rightarrow$ not}
% \item better at medium-range reordering
% \item outperforms phrase-based systems when translating between widely different languages, e.g. Chinese-English
% \end{itemize}
% 
% \item Disadvantages
% \begin{itemize}
% \item more disk usage --- translation model $\times$10 larger than phrase-based
% \item slower --- 0.5 - 2 sec/sent.\ for fastest configuration
% \item higher memory requirements --- more than 1GB RAM
% \end{itemize}
% 
% \end{itemize}


%Advantages
%   - can outperform phrase-based models when translating between widely different languages
%	- Chinese-English
%         - only consistent perforance with hierarchical model 
%   - better at medium range re-ordering
%        
%   - theoretical
%       - language is recursive
%       - embed clause within clause
%       - e.g. 
%        This is the house that Jack built.
%
%        This is the malt that lay in the house that Jack built.
%
%        This is the mouse that ate the malt that lay in the house that Jack built.
%
%        This is the cat that scared the mouse that ate the malt hat lay in the house that Jack built.
%
%        This is the dog that chased the cat that scared the mouse that ate the malt that lay in the house that Jack built.
%
%        This is the boy who loves the dog that chased the cat that scared the mouse that ate the malt that lay in the house that Jack built.


%%%%%%%%%%%%%%%%%%%%%%%%%%%%%%%%%%%%%%%%%%%%%%%%%%%%%%%%%%%%%%%%%%%%%%%%%%%%

%\slide{Hierarchical Model}
%
%Comparison with phrase-based model:
%
%\begin{table}[ht]
%\begin{tabular}{| c c | c c |}
%\hline 
%		& 	& Phrase-based & Hierarchical \\
%\hline
%BLEU (Europarl) & fr-en & 25.10 & 24.58 \\
%		& de-en & 18.11 & 17.99 \\
%		& es-en & 25.81 & 25.17 \\
%		& de-en & 18.11 & 17.99 \\
%		& cs-en & 18.00 & 17.86 \\
%\hline
%Phrase-table size & fr-en & 2.5GB & 20.0GB \\
%\hline
%Decoding time (sec) & per sentence & 2.27 & 6.45 \\
%		    & per word & 0.09 & 0.26 \\
%\hline
%\end{tabular}
%
%\end{table}
%}

%%%%%%%%%%%%%%%%%%%%%%%%%%%%%%%%%%%%%%%%%%%%%%%%%%%%%%%%%%%%%%%%%%%%%%%%%%%%

% \slide{Syntax-based Model Training with Moses}
% \vspace{20mm}
% \begin{itemize}
% \item Command line
% \begin{center}
% \begin{SaveVerbatim}{myverb} 
% train-model.perl -ghkm ...
% \end{SaveVerbatim}
% \colorbox{gray}{\BUseVerbatim{myverb}}
% \end{center}
% 
% \item Example rule from model
% \begin{center}
% \littlecode{\tiny [X][NP-SB] also wanted [X][ADJA] [X][NN] [X] ||| [X][NP-SB] wollten auch die [X][ADJA] [X][NN] [S-TOP] |||  ...}
% \end{center}
% \end{itemize}

%%%%%%%%%%%%%%%%%%%%%%%%%%%%%%%%%%%%%%%%%%%%%%%%%%%%%%%%%%%%%%%%%%%%%%%%%%%%

% \slide{Syntax-based Decoding with Moses}
% 
% \begin{itemize} \itemsep -1mm
% \item Command line   
% \vspace{-4mm}
% \begin{center}
% \begin{SaveVerbatim}{myverb}
% moses_chart -f moses.ini -i in.txt > out.txt
% \end{SaveVerbatim}
% \colorbox{gray}{\BUseVerbatim{myverb}}
% \end{center}
% \vspace{-4mm}
% (like hierarchical)
% 
% \item Advantage \vspace{-2mm}
% \begin{itemize}
% \item can use outside linguistic information 
% \item promises to solve important problems in SMT, e.g.\ long-range reordering
% \end{itemize}
% 
% \item Disadvantages \vspace{-2mm}
% \begin{itemize}
% %\item difficult to get right
% \item training slow and difficult to get right
% \item requires syntactic parse annotation
% 	\begin{itemize}
% 	\item syntactic parsers available only for some languages % need to be trained with costly, hand-annotated treebank
% 	\item not designed for machine translation
% 	\item unreliable
% 	\end{itemize}
% \end{itemize}
% \end{itemize}

%%%%%%%%%%%%%%%%%%%%%%%%%%%%%%%%%%%%%%%%%%%%%%%%%%%%%%%%%%%%%%%%%%%%%%%%%%%%

% \slide{Moses Tree Representation}
% \vspace{10mm}
% \begin{center}
% \Tree [.TOP [.NP [.NE Musharrafs ] [.ADJA letzter ] [.NN Akt ]  ]  [.PUNC  ? ]  ]  \\
% 
% \vspace{10mm}
% 
% \includegraphics[scale=1]{tree-xml.png} 
%  \end{center}

%
%Training and input data is embellished with parse information
%  - either the parse information is on the source side - tree-to-string
%  - on the target side - string-to-tree
%  - have parse information for both - tree-to-tree
%
%
%Advantages
%   - can use outside linguistic information
%Disadvantages
%   - difficult to get  right
%   - many still underperform other models
%   - parser information unreliable
%      - not for correct domain
%      - not available for many languages
%          - need to be trained with costly treebank
%      - not optmized or suited for machine translation
%
% do some slides on moses implementation

%%%%%%%%%%%%%%%%%%%%%%%%%%%%%%%%%%%%%%%%%%%%%%%%%%%%%%%%%%%%%%%%%%%%%%%%%%%%

\slide{Experiment Management System}
\begin{itemize} \itemsep -1mm
\item EMS automates the entire pipeline
\item One configuration file for all settings: record of all experimental details
\item Scheduler of individual steps in pipeline
\begin{itemize}
\item automatically keeps track of dependencies
%\item runs on single machine, multi-core machine, GridEngine cluster
\item parallel execution 
\item crash detection
\item automatic re-use of prior results
\end{itemize}
\item Fast to use
\begin{itemize}
\item set up a new experiment in minutes
\item set up a variation of an experiment in seconds
\end{itemize}

\item Disadvantage: not all Moses features are integrated

\end{itemize}

%%%%%%%%%%%%%%%%%%%%%%%%%%%%%%%%%%%%%%%%%%%%%%%%%%%%%%%%%%%%%%%%%%%%%%%%%%%%

\slide{How does the EMS work?}
\vspace{20mm}
\begin{itemize}
\item Write a configuration file
(typically by adapting an existing file)
\vspace{5mm}
\item Dryrun:
\vspace{-10mm}
\begin{center}
\littlecode{\normalsize experiment.perl -config config}
\end{center}
\vspace{5mm}
\item Execute:
\vspace{-10mm}
\begin{center}
\littlecode{\normalsize experiment.perl -config config -exec}
\end{center}
\end{itemize}

%%%%%%%%%%%%%%%%%%%%%%%%%%%%%%%%%%%%%%%%%%%%%%%%%%%%%%%%%%%%%%%%%%%%%%%%%%%%

\slide{}
\vspace*{-3.6ex}
%\begin{center} \vspace{-5mm}
\includegraphics[scale=1.5]{ems-agenda-composite.pdf}
%\end{center}

%%%%%%%%%%%%%%%%%%%%%%%%%%%%%%%%%%%%%%%%%%%%%%%%%%%%%%%%%%%%%%%%%%%%%%%%%%%%

% \slide{Web Interface}
% \begin{center}
% \includegraphics[scale=1]{web-interface-experiments.png}\\[5mm]
% List of experiments
% \end{center}

%%%%%%%%%%%%%%%%%%%%%%%%%%%%%%%%%%%%%%%%%%%%%%%%%%%%%%%%%%%%%%%%%%%%%%%%%%%%

% \slide{List of Runs}
% \begin{center}\vspace{-9mm}
% \includegraphics[scale=1]{web-interface-runs}
% \end{center}

%%%%%%%%%%%%%%%%%%%%%%%%%%%%%%%%%%%%%%%%%%%%%%%%%%%%%%%%%%%%%%%%%%%%%%%%%%%%

% \slide{Analysis: Basic Statistics}
% \begin{center}
% \includegraphics[scale=1]{analysis-stats.png}
% \end{center}
% \begin{itemize}
% \item Basic statistics
% \begin{itemize}
% \item n-gram precision
% \item evaluation metrics
% \item coverage of the input in corpus and translation model
% \item phrase segmentations used
% \end{itemize}
% \end{itemize}

%%%%%%%%%%%%%%%%%%%%%%%%%%%%%%%%%%%%%%%%%%%%%%%%%%%%%%%%%%%%%%%%%%%%%%%%%%%%

% \slide{Analysis: Unknown Words}
% \begin{center}
% grouped by count in test set\\[5mm]
% \includegraphics[scale=1]{analysis-unknown.png}
% \end{center}

%%%%%%%%%%%%%%%%%%%%%%%%%%%%%%%%%%%%%%%%%%%%%%%%%%%%%%%%%%%%%%%%%%%%%%%%%%%%

% \slide{Analysis: Output Annotation}
% \vspace{30mm}
% \begin{center}
% \includegraphics[scale=1.5]{analysis-bleu.png}\\[20mm]
% Color highlighting to indicate n-gram overlap with reference translation\\[5mm]
% darker bleu = word is part of larger n-gram match
% \end{center}

%%%%%%%%%%%%%%%%%%%%%%%%%%%%%%%%%%%%%%%%%%%%%%%%%%%%%%%%%%%%%%%%%%%%%%%%%%%%

% \slide{Analysis: Input Annotation}
% \vspace{10mm}
% \begin{center}
% \includegraphics[scale=1.5]{analysis-coverage.png}\\[20mm]
% \end{center}
% \begin{itemize}
% \item For each word and phrase, color coding and stats on
% \begin{itemize}
% \item number of occurrences in training corpus
% \item number of distinct translations in translation model
% \item entropy of conditional translation probability distribution $\phi(e|f)$ (normalized)
% \end{itemize}
% \end{itemize}

%%%%%%%%%%%%%%%%%%%%%%%%%%%%%%%%%%%%%%%%%%%%%%%%%%%%%%%%%%%%%%%%%%%%%%%%%%%%

% \slide{Analysis: Bilingual Concordancer}
% \vspace{-5mm}
% \begin{center}
% \includegraphics[scale=0.75]{biconcor1.png}\\[5mm]
% \includegraphics[scale=0.75]{biconcor2.png}\\
% translation of input phrase in training data context
% \end{center}

%%%%%%%%%%%%%%%%%%%%%%%%%%%%%%%%%%%%%%%%%%%%%%%%%%%%%%%%%%%%%%%%%%%%%%%%%%%%

% \slide{Analysis: Alignment}
% \vspace{30mm}
% \begin{center}
% \includegraphics[scale=1.5]{analysis-alignment.png}\\[10mm]
% Phrase alignment of the decoding process\\[5mm]
% (red border, interactive)
% \end{center}

%%%%%%%%%%%%%%%%%%%%%%%%%%%%%%%%%%%%%%%%%%%%%%%%%%%%%%%%%%%%%%%%%%%%%%%%%%%%

% \slide{Analysis: Tree Alignment}
% \vspace{15mm}
% \begin{center}
% \includegraphics[scale=1.2]{analysis-tree-alignment.png}\\[10mm]
% Uses nested boxes to indicate tree structure\\[3mm]
% (red border, yellow shaded spans in focus, interactive)\\[3mm]
% for syntax model, non-terminals are also shown
% \end{center}

%%%%%%%%%%%%%%%%%%%%%%%%%%%%%%%%%%%%%%%%%%%%%%%%%%%%%%%%%%%%%%%%%%%%%%%%%%%%

% \slide{Analysis: Comparison of 2 Runs}
% \begin{center}
% \includegraphics[scale=1]{analysis-comparison.png}\\[10mm]
% Different words are highlighted\\[3mm]
% sortable by most improvement, deterioration
% \end{center}

%%%%%%%%%%%%%%%%%%%%%%%%%%%%%%%%%%%%%%%%%%%%%%%%%%%%%%%%%%%%%%%%%%%%%%%%%%%%

\slide{}
\vspace{50mm}
\begin{center}
\Huge \bf Hands-On Session
\end{center}

%%%%%%%%%%%%%%%%%%%%%%%%%%%%%%%%%%%%%%%%%%%%%%%%%%%%%%%%%%%%%%%%%%%%%%%%%%%%
\includepdf[pages={1-17}]{Basics.pdf}

%%%%%%%%%%%%%%%%%%%%%%%%%%%%%%%%%%%%%%%%%%%%%%%%%%%%%%%%%%%%%%%%%%%%%%%%%%%%

\slide{Advanced Features}
\vspace{-5mm}
\textcolor{darkgrey}{
\begin{itemize} \itemsep -1mm
\item \currenttopic{Faster Training}
\item {Faster Decoding}
\item {Moses Server}
\item {Domain adaptation}
\item {Transliteration}
\item {Output formats}
\end{itemize}
}

%%%%%%%%%%%%%%%%%%%%%%%%%%%%%%%%%%%%%%%%%%%%%%%%%%%%%%%%%%%%%%%%%%%%%%%%%%%%

\slide{Advanced Features}
\vspace{-5mm}
\textcolor{darkgrey}{
\begin{itemize} \itemsep -1mm
%\item {How do I get started?}
%\item {Experiment Management System}
\item \currenttopic{Faster Training}
  \begin{itemize}
  \item Tokenization
  \item Tuning
  \item Alignment
  \item Phrase-table creation
  \item Language model creation
  \end{itemize}
\end{itemize}
\ldots
}

%%%%%%%%%%%%%%%%%%%%%%%%%%%%%%%%%%%%%%%%%%%%%%%%%%%%%%%%%%%%%%%%%%%%%%%%%%%%

\slide{Faster Training}

\begin{itemize} \itemsep 10mm
\vspace{10mm}
\item Run steps in parallel (that do not depend on each other)

\item {Multicore Parallelization}\\[4mm]
\begin{SaveVerbatim}{myverb} 
  .../train-model.perl -parallel
\end{SaveVerbatim}
\colorbox{gray}{\BUseVerbatim{myverb}}

\item EMS: \\[4mm]
\begin{SaveVerbatim}{myverb} 
  [TRAINING]
  parallel = yes
\end{SaveVerbatim}
\colorbox{gray}{\BUseVerbatim{myverb}}


\end{itemize}

%%%%%%%%%%%%%%%%%%%%%%%%%%%%%%%%%%%%%%%%%%%%%%%%%%%%%%%%%%%%%%%%%%%%%%%%%%%%

\slide{Advanced Features}
\vspace{-5mm}
\textcolor{darkgrey}{
\begin{itemize} \itemsep -1mm
%\item {How do I get started?}
%\item {Experiment Management System}
\item Faster Training
  \begin{itemize}
  \item \currenttopic{Tokenization}
  \item Tuning
  \item Alignment
  \item Phrase-table creation
  \item Language model creation
  \end{itemize}
\end{itemize}
\ldots
}

%%%%%%%%%%%%%%%%%%%%%%%%%%%%%%%%%%%%%%%%%%%%%%%%%%%%%%%%%%%%%%%%%%%%%%%%%%%%

\slide{Faster Training}

\begin{itemize} \itemsep 10mm

\item Multi-threaded tokenization

\item {Specify number of threads}\\[4mm]
\begin{SaveVerbatim}{myverb} 
 .../tokenizer.perl -threads NUM
\end{SaveVerbatim}
\colorbox{gray}{\BUseVerbatim{myverb}}

\item EMS: \\[4mm]
\begin{SaveVerbatim}{myverb} 
 input-tokenizer = "$moses-script-dir/tokenizer/tokenizer.perl 
		    -threads NUM " 
\end{SaveVerbatim}
\colorbox{gray}{\BUseVerbatim{myverb}}
  
  
\end{itemize}


%%%%%%%%%%%%%%%%%%%%%%%%%%%%%%%%%%%%%%%%%%%%%%%%%%%%%%%%%%%%%%%%%%%%%%%%%%%%

\slide{Advanced Features}
\vspace{-5mm}
\textcolor{darkgrey}{
\begin{itemize} \itemsep -1mm
%\item {How do I get started?}
%\item {Experiment Management System}
\item Faster Training
  \begin{itemize}
  \item Tokenization
  \item \currenttopic{Tuning}
  \item Alignment
  \item Phrase-table creation
  \item Language model creation
  \end{itemize}
\end{itemize}
\ldots
}

%%%%%%%%%%%%%%%%%%%%%%%%%%%%%%%%%%%%%%%%%%%%%%%%%%%%%%%%%%%%%%%%%%%%%%%%%%%%

\slide{Faster Training}
\vspace{10mm}
\begin{itemize} \itemsep 10mm
\item Multi-threaded tokenization

\item {Specify number of threads}\\[4mm]
\begin{SaveVerbatim}{myverb} 
  .../mert -threads NUM
\end{SaveVerbatim}
\colorbox{gray}{\BUseVerbatim{myverb}}

\item EMS:\\[4mm]
\begin{SaveVerbatim}{myverb} 
  tuning-settings = "-threads NUM" 
\end{SaveVerbatim}
\colorbox{gray}{\BUseVerbatim{myverb}}

\end{itemize}


%%%%%%%%%%%%%%%%%%%%%%%%%%%%%%%%%%%%%%%%%%%%%%%%%%%%%%%%%%%%%%%%%%%%%%%%%%%%

\slide{Advanced Features}
\vspace{-5mm}
\textcolor{darkgrey}{
\begin{itemize} \itemsep -1mm
%\item {How do I get started?}
%\item {Experiment Management System}
\item Faster Training
  \begin{itemize}
  \item Tokenization
  \item Tuning
  \item \currenttopic{Alignment}
  \item Phrase-table creation
  \item Language model creation
  \end{itemize}
\end{itemize}
\ldots
}

%%%%%%%%%%%%%%%%%%%%%%%%%%%%%%%%%%%%%%%%%%%%%%%%%%%%%%%%%%%%%%%%%%%%%%%%%%%%

\slide{Faster Training}

\begin{itemize} \itemsep -1mm

\item {Word Alignment}
\item {Multi-threaded}
  \begin{itemize}
  \item    Use MGIZA, not GIZA++
  \begin{center}
    \littlecode{.../train-model.perl -mgiza -mgiza-cpus NUM} 
  \end{center}      
  EMS: 
  \begin{center}
    \littlecode{training-options = " -mgiza -mgiza-cpus NUM " } 
  \end{center}      
  \end{itemize}

\item {On: memory-limited machines}
  \begin{itemize}
  \item snt2cooc program requires 6GB+ memory
  \item Reimplementation uses 10MB, but take longer to run
  \begin{center}
    \littlecode{.../train-model.perl -snt2cooc snt2cooc.pl} 
  \end{center}      
  EMS:
  \begin{center}
    \littlecode{training-options = "-snt2cooc snt2cooc.pl"}
  \end{center}      

  \end{itemize}
\end{itemize}
       

%%%%%%%%%%%%%%%%%%%%%%%%%%%%%%%%%%%%%%%%%%%%%%%%%%%%%%%%%%%%%%%%%%%%%%%%%%%%

\slide{Advanced Features}
\vspace{-5mm}
\textcolor{darkgrey}{
\begin{itemize} \itemsep -1mm
%\item {How do I get started?}
%\item {Experiment Management System}
\item Faster Training
  \begin{itemize}
  \item Tokenization
  \item Tuning
  \item Alignment
  \item \currenttopic{Phrase-table creation}
  \item Language model creation
  \end{itemize}
\end{itemize}
\ldots
}


%%%%%%%%%%%%%%%%%%%%%%%%%%%%%%%%%%%%%%%%%%%%%%%%%%%%%%%%%%%%%%%%%%%%%%%%%%%%

\slide{Faster Training}
\vspace{30mm}
\begin{itemize} \itemsep -1mm

\item {Phrase-Table Extraction}
  \begin{itemize}
  \item Split training data into NUM equal parts
  \item Extract concurrently
  \end{itemize}
  \begin{center}
    \littlecode{.../train-model.perl -cores NUM}
  \end{center}      
\end{itemize}

%%%%%%%%%%%%%%%%%%%%%%%%%%%%%%%%%%%%%%%%%%%%%%%%%%%%%%%%%%%%%%%%%%%%%%%%%%%%

\slide{Faster Training}
\vspace{10mm}
\begin{itemize} \itemsep -1mm
\item {Sorting}
  \begin{itemize}
  \item Rely heavily on Unix 'sort' command
    \item may take 50\%+ of translation model build time 
  \item Need to optimize for
     \begin{itemize}
      \item speed
      \item disk usage
     \end{itemize}

  \item Dependent on
    \begin{itemize}
    \item      sort version
    \item      Unix version
    \item      available memory
    \end{itemize}
  \end{itemize}
\end{itemize}

%%%%%%%%%%%%%%%%%%%%%%%%%%%%%%%%%%%%%%%%%%%%%%%%%%%%%%%%%%%%%%%%%%%%%%%%%%%%

\slide{Faster Training}
\vspace{-5mm}
  \begin{itemize} \itemsep -1mm 
  \item Plain sorted\\[2mm]
  \begin{SaveVerbatim}{myverb}
 sort < extract.txt > extract.sorted.txt
  \end{SaveVerbatim}
  \colorbox{gray}{\BUseVerbatim{myverb}}

  \item Optimized for large server\\[2mm]
  \begin{SaveVerbatim}{myverb}
 sort --buffer-size 10G --parallel 5
      --batch-size 253 --compress-program [gzip/pigz] ...
  \end{SaveVerbatim}
  \colorbox{gray}{\BUseVerbatim{myverb}}
    \begin{itemize}
    \item Use 10GB of RAM --- the more the better
    \item 5 CPUs --- the more the better
    \item merge\-sort at most 253 files
    \item compress intermediate files --- less disk i/o
    \end{itemize}

\item In Moses:\\[2mm]
    \begin{SaveVerbatim}{myverb}
 .../train-model.perl -sort-buffer-size 10G -sort-parallel 5 
      -sort-batch-size 253 -sort-compress pigz 
    \end{SaveVerbatim}
    \colorbox{gray}{\BUseVerbatim{myverb}}

  \end{itemize}


%%%%%%%%%%%%%%%%%%%%%%%%%%%%%%%%%%%%%%%%%%%%%%%%%%%%%%%%%%%%%%%%%%%%%%%%%%%%

\slide{Advanced Features}
\vspace{-5mm}
\textcolor{darkgrey}{
\begin{itemize} \itemsep -1mm
%\item {How do I get started?}
%\item {Experiment Management System}
\item Faster Training
  \begin{itemize}
  \item Tokenization
  \item Tuning
  \item Alignment
  \item Phrase-table creation
  \item \currenttopic{Language model creation}
  \end{itemize}
\end{itemize}
\ldots
}

%%%%%%%%%%%%%%%%%%%%%%%%%%%%%%%%%%%%%%%%%%%%%%%%%%%%%%%%%%%%%%%%%%%%%%%%%%%%
\slide{KENLM Training}
\vspace{5mm}
\begin{itemize}
\item Can train very large language models with limited RAM\\
(on disk streaming)\\[5mm]
\begin{SaveVerbatim}{myverb} 
lmplz -o [order] -S [memory] < text > text.lm
\end{SaveVerbatim}
\colorbox{gray}{\BUseVerbatim{myverb}}

\item \littlecode{-o order} = n-gram order
\item \littlecode{-S memory} = How much memory to use.
	      \begin{itemize}
		\item \littlecode{NUM\%} = percentage of physical memory \vspace{2mm}
		\item \littlecode{NUM[b/K/M/G/T]} = specified amount in bytes, kilo bytes, etc.
	      \end{itemize}
\end{itemize}


%%%%%%%%%%%%%%%%%%%%%%%%%%%%%%%%%%%%%%%%%%%%%%%%%%%%%%%%%%%%%%%%%%%%%%%%%%%%

\slide{Advanced Features}
\vspace{-5mm}
\textcolor{darkgrey}{
\begin{itemize} \itemsep -1mm
\item {Faster Training}
\item \currenttopic{Faster Decoding}
\item {Moses Server}
\item {Domain adaptation}
\item {Transliteration}
\item {Output formats}
\end{itemize}
}


%%%%%%%%%%%%%%%%%%%%%%%%%%%%%%%%%%%%%%%%%%%%%%%%%%%%%%%%%%%%%%%%%%%%%%%%%%%%

\slide{Advanced Features}
\vspace{-5mm}
\textcolor{darkgrey}{
\begin{itemize} \itemsep -1mm
%\item {How do I get started?}
%\item {Experiment Management System}
\item {Faster Training}
\item \currenttopic{Faster Decoding}
  \begin{itemize}
  \item Multi-threading
  \item Speed vs. Memory
  \item Speed vs. Quality
  \end{itemize}
\end{itemize}
\ldots
}

%%%%%%%%%%%%%%%%%%%%%%%%%%%%%%%%%%%%%%%%%%%%%%%%%%%%%%%%%%%%%%%%%%%%%%%%%%%%

\slide{Advanced Features}
\vspace{-5mm}
\textcolor{darkgrey}{
\begin{itemize} \itemsep -1mm
%\item {How do I get started?}
%\item {Experiment Management System}
\item {Faster Training}
\item {Faster Decoding}
  \begin{itemize}
  \item \currenttopic{Multi-threading}
  \item Speed vs. Memory
  \item Speed vs. Quality
  \end{itemize}
\end{itemize}
\ldots
}

%%%%%%%%%%%%%%%%%%%%%%%%%%%%%%%%%%%%%%%%%%%%%%%%%%%%%%%%%%%%%%%%%%%%%%%%%%%%

\slide{Faster Decoding}

\vspace{15mm}
\begin{itemize}
\item Multi-threaded decoding 
\begin{center}
\littlecode{.../moses --threads NUM}
\end{center}
\item Easy speed-up

\end{itemize}


%%%%%%%%%%%%%%%%%%%%%%%%%%%%%%%%%%%%%%%%%%%%%%%%%%%%%%%%%%%%%%%%%%%%%%%%%%%%

\slide{Advanced Features}
\vspace{-5mm}
\textcolor{darkgrey}{
\begin{itemize} \itemsep -1mm
%\item {How do I get started?}
%\item {Experiment Management System}
\item {Faster Training}
\item {Faster Decoding}
  \begin{itemize}
  \item {Multi-threading}
  \item \currenttopic{Speed vs. Memory}
  \item Speed vs. Quality
  \end{itemize}
\end{itemize}
\ldots
}

%%%%%%%%%%%%%%%%%%%%%%%%%%%%%%%%%%%%%%%%%%%%%%%%%%%%%%%%%%%%%%%%%%%%%%%%%%%%

%\slide{Speed vs. Memory Use}
%\begin{center} 
%\includegraphics[scale=1.4]{less-memory.pdf}
%\end{center}

%%%%%%%%%%%%%%%%%%%%%%%%%%%%%%%%%%%%%%%%%%%%%%%%%%%%%%%%%%%%%%%%%%%%%%%%%%%%

\slide{Speed vs. Memory Use}
\begin{tabular}{p{10cm}c}
\vspace{-11cm}
Typical Europarl file sizes:
\begin{itemize} \itemsep -1mm
  \item Language model \vspace{-3mm}
	\begin{itemize}
  	\item  170 MB (trigram)
	\item 412 MB (5-gram)
	\end{itemize}
  \item Phrase table \vspace{-3mm}
	\begin{itemize}
  	\item  11GB
	\end{itemize}
  \item Lexicalized reordering \vspace{-3mm}
	\begin{itemize}
  	\item  9.4GB
	\end{itemize}
   \item[$\rightarrow$] total = 20.8 GB
\end{itemize}
&
\includegraphics[scale=1.4]{less-memory-europarl.pdf}
\end{tabular}

%%%%%%%%%%%%%%%%%%%%%%%%%%%%%%%%%%%%%%%%%%%%%%%%%%%%%%%%%%%%%%%%%%%%%%%%%%%%

\slide{Speed vs. Memory Use}
\vspace{10mm}
\begin{tabular}{p{13cm}c}
\vspace{-7cm}
\begin{itemize}
\item Load into memory
	\begin{itemize}
	\item long load time
	\item large memory usage
	\item fast decoding
	\end{itemize}
\end{itemize}
& \includegraphics[scale=0.8]{less-memory-europarl.pdf} \\[1cm]
\vspace{-4.5cm}
\begin{itemize}
\item Load-on-demand
	\begin{itemize}
  	\item store indexed model on disk
	\item binary format
	\item minimal start-up time, memory usage
	\item slower decoding
	\end{itemize}
\end{itemize}
&  \includegraphics[scale=0.8]{less-memory-europarl-on-disk.pdf}
\end{tabular}

%%%%%%%%%%%%%%%%%%%%%%%%%%%%%%%%%%%%%%%%%%%%%%%%%%%%%%%%%%%%%%%%%%%%%%%%%%%%

\slide{Binary Phrase Tables}
Create:
\begin{center}
\begin{SaveVerbatim}{myverb} 
  ./CreateOnDiskPt 1 1 4 100 2 pt.txt out.folder
\end{SaveVerbatim}
\colorbox{gray}{\BUseVerbatim{myverb}}
\end{center}

Change ini file: \\[-6mm]
\begin{center}
\begin{SaveVerbatim}{myverb} 
[feature]
PhraseDictionaryBinary name=TranslationModel0 \
   table-limit=20 \ num-features=4 \
   path=/.../phrase-table
\end{SaveVerbatim}
\colorbox{gray}{\BUseVerbatim{myverb}}
\end{center}

%%%%%%%%%%%%%%%%%%%%%%%%%%%%%%%%%%%%%%%%%%%%%%%%%%%%%%%%%%%%%%%%%%%%%%%%%%%%

\slide{Lexical Reordering Table}

Create:
\begin{center}
\begin{SaveVerbatim}{myverb} 
  export LC_ALL=C \ 
  processLexicalTable -in r-t.txt -out out.file
\end{SaveVerbatim}
\colorbox{gray}{\BUseVerbatim{myverb}}
\end{center}

Change ini file: \\[-6mm]
\begin{center}
automatically detected
\end{center}

%%%%%%%%%%%%%%%%%%%%%%%%%%%%%%%%%%%%%%%%%%%%%%%%%%%%%%%%%%%%%%%%%%%%%%%%%%%%

\slide{Compact Phrase Table}
\begin{itemize}
\item Memory-efficient data structure
\begin{itemize}
\item phrase table 6--7 times smaller than on-disk binary table
\item lexical reordering table 12--15 times smaller than on-disk binary table
\item Fastest phrase-table implementation
\end{itemize}
\item Create with
\begin{center}
\begin{SaveVerbatim}{myverb} 
processPhraseTableMin
processLexicalTableMin
\end{SaveVerbatim}
\colorbox{gray}{\BUseVerbatim{myverb}}
\end{center}
\item Specify with 
\begin{center}
\begin{SaveVerbatim}{myverb} 
PhraseDictionaryCompact
\end{SaveVerbatim}
\colorbox{gray}{\BUseVerbatim{myverb}}
\end{center}
\end{itemize}

%%%%%%%%%%%%%%%%%%%%%%%%%%%%%%%%%%%%%%%%%%%%%%%%%%%%%%%%%%%%%%%%%%%%%%%%%%%%

\slide{KENLM}
\vspace{10mm}
\begin{itemize}
\item Developed by Kenneth Heafield (CMU / Edinburgh / Stanford / Bloomberg)
\item Fastest and smallest language model implementation
\item Compile from LM trained with SRILM\\[5mm]
\begin{SaveVerbatim}{myverb} 
build_binary model.lm model.binlm
\end{SaveVerbatim}
\colorbox{gray}{\BUseVerbatim{myverb}}
\item Specify in decoder\\[5mm]
\begin{SaveVerbatim}{myverb} 
[feature]
KENLM name=LM0 factor=0 path=/.../model.binlm order=5
\end{SaveVerbatim}
\colorbox{gray}{\BUseVerbatim{myverb}}
\end{itemize}

%%%%%%%%%%%%%%%%%%%%%%%%%%%%%%%%%%%%%%%%%%%%%%%%%%%%%%%%%%%%%%%%%%%%%%%%%%%%

\slide{Advanced Features}
\vspace{-5mm}
\textcolor{darkgrey}{
\begin{itemize} \itemsep -1mm
%\item {How do I get started?}
%\item {Experiment Management System}
\item {Faster Training}
\item {Faster Decoding}
  \begin{itemize}
  \item {Multi-threading}
  \item Speed vs. Memory
  \item \currenttopic{Speed vs. Quality}
  \end{itemize}
\end{itemize}
\ldots
}

%%%%%%%%%%%%%%%%%%%%%%%%%%%%%%%%%%%%%%%%%%%%%%%%%%%%%%%%%%%%%%%%%%%%%%%%%%%%

\slide{Speed vs. Quality}
\vspace{5mm}
\begin{center} 
\includegraphics[scale=1.4]{quality-vs-speed.pdf}\vspace{-20mm}
\end{center}

%%%%%%%%%%%%%%%%%%%%%%%%%%%%%%%%%%%%%%%%%%%%%%%%%%%%%%%%%%%%%%%%%%%%%%%%%%%%

\slide{Speed vs. Quality}
\begin{center} 
\includegraphics[scale=0.95]{decoding-step5.pdf}
\end{center}
\begin{itemize} \itemsep -1mm
\item Decoder search creates very large number of partial translations ("hypotheses")
\item Decoding time $\sim$ number of hypotheses created
\item Translation quality $\sim$ number of hypothesis created
\end{itemize}

%%%%%%%%%%%%%%%%%%%%%%%%%%%%%%%%%%%%%%%%%%%%%%%%%%%%%%%%%%%%%%%%%%%%%%%%%%%%

\slide{Hypothesis Stacks}
\begin{center} 
\includegraphics[scale=1]{hypothesis-stacks-fw.pdf}
\end{center}
\vspace{-2mm}
\begin{itemize} \itemsep -2mm
\item Phrase-based: One stack per number of input words covered
\item Number of hypothesis created = \\
sentence length $\times$ stack size $\times$ applicable translation options
\end{itemize}

%%%%%%%%%%%%%%%%%%%%%%%%%%%%%%%%%%%%%%%%%%%%%%%%%%%%%%%%%%%%%%%%%%%%%%%%%%%%

\slide{Pruning Parameters}
\begin{itemize}
\item Regular beam search
\begin{itemize}
\item \littlecode{--stack NUM} max. number of hypotheses contained in each stack
\item \littlecode{--ttable-limit NUM} max. num. of translation options per input phrase
\vspace{2mm}
\item search time roughly linear with respect to each number
\end{itemize}
\item Cube pruning\\
(fixed number of hypotheses are added to each stack)
\begin{itemize}
\item \littlecode{--search-algorithm 1} turns on cube pruning
\item \littlecode{--cube-pruning-pop-limit NUM} number of hypotheses added to each stack
\vspace{2mm}
\item search time roughly linear with respect to pop limit
\vspace{2mm}
\item note: stack size and translation table limit have little impact in speed
\end{itemize}
\end{itemize}


%%%%%%%%%%%%%%%%%%%%%%%%%%%%%%%%%%%%%%%%%%%%%%%%%%%%%%%%%%%%%%%%%%%%%%%%%%%%

\slide{Syntax Hypothesis Stacks}
\begin{center} 
\includegraphics[scale=1.5]{chart-stacks.pdf}
\end{center}
\vspace{-5mm}
\begin{itemize} \itemsep -2mm
\item One stack per input word span
\item Number of hypothesis created = \\
sentence length$^2$ $\times$  number of hypotheses added to each stack\\
\littlecode{--cube-pruning-pop-limit NUM} number of hypotheses added to each stack
\end{itemize}

%%%%%%%%%%%%%%%%%%%%%%%%%%%%%%%%%%%%%%%%%%%%%%%%%%%%%%%%%%%%%%%%%%%%%%%%%%%%

\slide{Advanced Features}
\vspace{-5mm}
\textcolor{darkgrey}{
\begin{itemize} \itemsep -1mm
\item {Faster Training}
\item {Faster Decoding}
\item \currenttopic{Moses Server}
\item {Domain adaptation}
\item {Transliteration}
\item {Output formats}
\end{itemize}
}

%%%%%%%%%%%%%%%%%%%%%%%%%%%%%%%%%%%%%%%%%%%%%%%%%%%%%%%%%%%%%%%%%%%%%%%%%%%%

\slide{Moses Server}
\begin{itemize} \itemsep -1mm
\item Moses command line:\\[3mm]
\begin{SaveVerbatim}{myverb} 
  .../moses -f [ini] < [input file] > [output file]
\end{SaveVerbatim}
\colorbox{gray}{\BUseVerbatim{myverb}}
    \begin{itemize}
      \item Not practical for commercial use
    \end{itemize}


\item Moses Server:\\[3mm]
    \begin{SaveVerbatim}{myverb} 
 .../mosesserver -f [ini] --server-port [PORT] --server-log [LOG]
    \end{SaveVerbatim}
    \colorbox{gray}{\BUseVerbatim{myverb}}
    \begin{itemize}
      \item Accept HTTP input. XML SOAP format
    \end{itemize}

\item Client:
    \begin{itemize}
      \item Communicate via http
      \item Example clients in Java and Perl
      \item Write your own client
      \item Integrate into your own application
    \end{itemize}
    
\end{itemize}

%%%%%%%%%%%%%%%%%%%%%%%%%%%%%%%%%%%%%%%%%%%%%%%%%%%%%%%%%%%%%%%%%%%%%%%%%%%%


\slide{Advanced Features}
\vspace{-5mm}
\textcolor{darkgrey}{
\begin{itemize} \itemsep -1mm
\item {Faster Training}
\item {Faster Decoding}
\item {Moses Server}
\item \currenttopic{Domain adaptation}
  \begin{itemize}\vspace{-4mm}
  \item Train everything together
  \item Secondary phrase table
  \item Domain indicator features
  \item Interpolated language models
  \end{itemize}
\item {Transliteration}
\item {Output formats}
\end{itemize}
}

%%%%%%%%%%%%%%%%%%%%%%%%%%%%%%%%%%%%%%%%%%%%%%%%%%%%%%%%%%%%%%%%%%%%%%%%%%%%

\slide{Data}
\vspace{15mm}
\begin{itemize}
\item Parallel corpora $\rightarrow$ translation model
\begin{itemize}
\item sentence-aligned translated texts
\item translation memories are parallel corpora
\item dictionaries are parallel corpora
\end{itemize}
\item Monolingual corpora $\rightarrow$ language model
\begin{itemize}
\item text in the target language
\item billions of words easy to handle
\end{itemize}
\end{itemize}

%%%%%%%%%%%%%%%%%%%%%%%%%%%%%%%%%%%%%%%%%%%%%%%%%%%%%%%%%%%%%%%%%%%%%%%%%%%%

\slide{Domain Adaptation}
\vspace{10mm}
\begin{itemize}
\item The more data, the better
\item The more in-domain data, the better\\
(even in-domain monolingual data very valuable)
%\item Multiple models 
%\begin{itemize}
%\item train a translation model for each domain corpus
%\item train a language model for each domain corpus
%\item use all, tune weights for each model
%\item alternative: interpolate language model
%\end{itemize}
\item Always tune towards target domain
\end{itemize}

%%%%%%%%%%%%%%%%%%%%%%%%%%%%%%%%%%%%%%%%%%%%%%%%%%%%%%%%%%%%%%%%%%%%%%%%%%%%

\slide{Advanced Features}
\vspace{-5mm}
\textcolor{darkgrey}{
\begin{itemize} \itemsep -1mm
%\item {How do I get started?}
%\item {Experiment Management System}
\item {Faster Training}
\item {Faster Decoding}
\item {Moses Server}
\item {Data and domain adaptation}
  \begin{itemize}\vspace{-5mm}
  \item \currenttopic{Train everything together}
  \item Secondary phrase table
\item Domain indicator features
\item Interpolated language models
%\item TM-MT integration
  \end{itemize}
\end{itemize}
}

%%%%%%%%%%%%%%%%%%%%%%%%%%%%%%%%%%%%%%%%%%%%%%%%%%%%%%%%%%%%%%%%%%%%%%%%%%%%

\slide{Default: Train Everything Together}
\vspace{20mm}
\begin{itemize} \itemsep 10mm
 
\item Easy to implement
  \begin{itemize}
  \item Concatenate new data with existing data
  \item Retrain
  \end{itemize}
\item Disadvantages: 
  \begin{itemize}
  \item Slower training for large amount of data
  \item Cannot weight old and new data separately
  \end{itemize}
\end{itemize}

%%%%%%%%%%%%%%%%%%%%%%%%%%%%%%%%%%%%%%%%%%%%%%%%%%%%%%%%%%%%%%%%%%%%%%%%%%%%

\slide{Default: Train Everything Together}
\vspace{-1mm}
Specification in EMS:
\begin{itemize} \itemsep 5mm
\item Phrase-table\\[4mm] \small
    \begin{SaveVerbatim}{myverb} 
      [CORPUS]
      [CORPUS:in-domain]
      raw-stem = ....    
      [CORPUS:background]
      raw-stem = ....
    \end{SaveVerbatim}
    \colorbox{gray}{\BUseVerbatim{myverb}}
\item LM\\[4mm] \small
   \begin{SaveVerbatim}{myverb} 
      [LM]
      [LM:in-domain]
      raw-corpus = ....
      [LM:background]
      raw-corpus = ....
    \end{SaveVerbatim}
    \colorbox{gray}{\BUseVerbatim{myverb}}
\end{itemize}

%%%%%%%%%%%%%%%%%%%%%%%%%%%%%%%%%%%%%%%%%%%%%%%%%%%%%%%%%%%%%%%%%%%%%%%%%%%%

\slide{Advanced Features}
\vspace{-5mm}
\textcolor{darkgrey}{
\begin{itemize} \itemsep -1mm
%\item {How do I get started?}
%\item {Experiment Management System}
\item {Faster Training}
\item {Faster Decoding}
\item {Moses Server}
\item {Data and domain adaptation}
  \begin{itemize}\vspace{-5mm}
  \item {Train everything together}
  \item \currenttopic{Secondary phrase table}
\item Domain indicator features
\item Interpolated language models
%\item TM-MT integration
  \end{itemize}
\end{itemize}
}

%%%%%%%%%%%%%%%%%%%%%%%%%%%%%%%%%%%%%%%%%%%%%%%%%%%%%%%%%%%%%%%%%%%%%%%%%%%%

%\slide{TM-MT Integration}
%\begin{itemize}
%\item Input sentence: \vspace{-5mm}
%\begin{center}
%\example{The second paragraph of Article \highlightOrange{21} is deleted .}
%\end{center}
%\item Fuzzy match in translation memory: \vspace{-5mm}
%\begin{center}
%\example{The second paragraph of Article \highlight{5} is deleted .}\\
%=\\
%\example{\highlightGreen{{\`A} l' article} \highlight{5} \highlightGreen{, le texte du deuxi{\'e}me alin{\'e}a est supprim{\'e} .}}
%\end{center}
%\item[] \textcolor{darkgreen}{\bf Output word(s) taken from the target TM}  \vspace{-5mm}
%\item[] \textcolor{darkorange}{\bf Input word(s) that still need to be translated by SMT}
%\end{itemize}

%%%%%%%%%%%%%%%%%%%%%%%%%%%%%%%%%%%%%%%%%%%%%%%%%%%%%%%%%%%%%%%%%%%%%%%%%%%%


%\slide{TM-MT Integration}

%\begin{itemize} \itemsep -1mm
%\item Translation memory-style fuzzy match
%  \begin{itemize}
%  \item For hierarchical decoding
%  \item Create long translation rule 'templates'
%  \item Best for use with parallel corpus with lots of repetition
%  \end{itemize}

%\item Add TM and word alignment as a special phrase table
%  \begin{itemize}
%    \item Use in addition to normal phrase table
%  \end{itemize}
%  \begin{SaveVerbatim}{myverb} 
%   [ttable-file]
%   11 0 0 3 source-corpus;target-corpus;word-alignment 
%   2 0 0 3 phrase table
%   6 0 0 3 glue-rules
%  \end{SaveVerbatim}
%  \colorbox{gray}{\BUseVerbatim{myverb}}

%\end{itemize}

%%%%%%%%%%%%%%%%%%%%%%%%%%%%%%%%%%%%%%%%%%%%%%%%%%%%%%%%%%%%%%%%%%%%%%%%%%%%

\slide{Secondary Phrase Table}
\begin{itemize} \itemsep -1mm
\item Train initial phrase table and LM on baseline data

\item Train secondary phrase table and LM new/in-domain data

\item Use both in Moses
  \begin{itemize}

  \item Secondary phrase table
  \begin{SaveVerbatim}{myverb} 
    [feature]
    PhraseDictionaryMemory path=.../path.1
    PhraseDictionaryMemory path=.../path.2
    
    [mapping]
    0 T 0
    1 T 1
  \end{SaveVerbatim}
  \colorbox{gray}{\BUseVerbatim{myverb}}

  \end{itemize}
\end{itemize}

%%%%%%%%%%%%%%%%%%%%%%%%%%%%%%%%%%%%%%%%%%%%%%%%%%%%%%%%%%%%%%%%%%%%%%%%%%%%

\slide{Secondary Phrase Table}
\begin{itemize} \itemsep -1mm
  \item
  \begin{itemize}
  \item Secondary LM\\[4mm]
  \begin{SaveVerbatim}{myverb} 
    [feature]
    KENLM path=.../path.1
    KENLM path=.../path.2
  \end{SaveVerbatim}
  \colorbox{gray}{\BUseVerbatim{myverb}}

  \end{itemize}

  \item Can give different weights for primary and secondary tables
  \item Not integrated into the EMS
\end{itemize}

%%%%%%%%%%%%%%%%%%%%%%%%%%%%%%%%%%%%%%%%%%%%%%%%%%%%%%%%%%%%%%%%%%%%%%%%%%%%

\slide{Secondary Phrase Table}
\small
\begin{itemize} \itemsep -1mm
\item Terminology/Glossary database
  \begin{itemize}
    \item fixed translation
    \item per client, project, etc
  \end{itemize}
\item Primary phrase table
  \begin{itemize}
    \item backoff to 'normal' phrase-table if no glossary term

  \begin{SaveVerbatim}{myverb} 
    [feature]
    PhraseDictionaryMemory path=.../glossary
    PhraseDictionaryMemory path=.../normal.phrase.table
    
    [mapping]
    0 T 0
    1 T 1
    
    [decoding-graph-backoff]
    0
    1
  \end{SaveVerbatim}
  \colorbox{gray}{\BUseVerbatim{myverb}}

    \end{itemize}
\end{itemize}
\normalsize 
%%%%%%%%%%%%%%%%%%%%%%%%%%%%%%%%%%%%%%%%%%%%%%%%%%%%%%%%%%%%%%%%%%%%%%%%%%%%
% 
% \slide{Incremental training}
% \begin{itemize}
% \item Retrain everything
% \item Secondary phrase table
% \item \currenttopic{Incremental GIZA++ and dynamic suffix arrays}
% \item TM-MT integration
% 
% \end{itemize}
% 
% %%%%%%%%%%%%%%%%%%%%%%%%%%%%%%%%%%%%%%%%%%%%%%%%%%%%%%%%%%%%%%%%%%%%%%%%%%%%
% 
% \slide{Incremental GIZA++ and Dynamic Suffix Arrays}
% 
% \begin{itemize}
% \item Don't extract phrase table
% \item Store entire parallel corpus in memory
% 	\\ Suffix Array
% \item Add new parallel data to suffix array
% \\
% \item Need word alignment
%   \\ Use customized version of GIZA++
%   \\ Reuse word-alignment model from primary parallel data
% 
% \item Bleeding edge. Not integrated into EMS
% \end{itemize}


%%%%%%%%%%%%%%%%%%%%%%%%%%%%%%%%%%%%%%%%%%%%%%%%%%%%%%%%%%%%%%%%%%%%%%%%%%%%

\slide{Advanced Features}
\vspace{-5mm}
\textcolor{darkgrey}{
\begin{itemize} \itemsep -1mm
%\item {How do I get started?}
%\item {Experiment Management System}
\item {Faster Training}
\item {Faster Decoding}
\item {Moses Server}
\item {Data and domain adaptation}
  \begin{itemize} \vspace{-5mm}
  \item {Train everything together}
  \item {Secondary phrase table}
  \item \currenttopic{Domain indicator features}
\item Interpolated language models
%  \item TM-MT integration
  \end{itemize}
\end{itemize}
}

%%%%%%%%%%%%%%%%%%%%%%%%%%%%%%%%%%%%%%%%%%%%%%%%%%%%%%%%%%%%%%%%%%%%%%%%%%%%

\slide{Domain Indicator Features}
\vspace{15mm}
\begin{itemize} \itemsep 5mm
\item One translation model
\item Flag each phrase pair's origin
\begin{itemize}
\item indicator: binary flag if it occurs in specific domain
\item ratio: how often it occurs in specific domain relative to all
\item subset: similar to indicator, but if in multiple domains, marked with multiple-domain feature
\end{itemize}
\item In EMS:\\[4mm] \small
\begin{SaveVerbatim}{myverb}
  [TRAINING]
  domain-features = "indicator"
\end{SaveVerbatim}
\colorbox{gray}{\BUseVerbatim{myverb}}
\end{itemize}

%%%%%%%%%%%%%%%%%%%%%%%%%%%%%%%%%%%%%%%%%%%%%%%%%%%%%%%%%%%%%%%%%%%%%%%%%%%%

\slide{Advanced Features}
\vspace{-5mm}
\textcolor{darkgrey}{
\begin{itemize} \itemsep -1mm
%\item {How do I get started?}
%\item {Experiment Management System}
\item {Faster Training}
\item {Faster Decoding}
\item {Moses Server}
\item {Data and domain adaptation}
  \begin{itemize} \vspace{-5mm}
  \item {Train everything together}
  \item {Secondary phrase table}
  \item Domain indicator features
\item \currenttopic{Interpolated language models}
%  \item TM-MT integration
  \end{itemize}
\end{itemize}
}

%%%%%%%%%%%%%%%%%%%%%%%%%%%%%%%%%%%%%%%%%%%%%%%%%%%%%%%%%%%%%%%%%%%%%%%%%%%%

\slide{Interpolated Language Models}
\vspace{5mm}
\begin{itemize} \itemsep 5mm
\item Train one language model per corpus
\item Combine them by weighting each according to its importance
\begin{itemize}
\item weights obtained by optimizing perplexity\\ of resulting language model on tuning set\\
(not the same as machine translation quality) \vspace{2mm}
\item models are linearly combined
\end{itemize}
\item EMS provides a section {\tt [INTERPOLATED-LM]} that needs to be commented out
\item Alternative: use multiple language models\\
(disadvantage: larger process, slower)
\end{itemize}

%%%%%%%%%%%%%%%%%%%%%%%%%%%%%%%%%%%%%%%%%%%%%%%%%%%%%%%%%%%%%%%%%%%%%%%%%%%%

%\slide{Advanced Features}
%\vspace{-5mm}
%\textcolor{darkgrey}{
%\begin{itemize} \itemsep -1mm
%%\item {How do I get started?}
%%\item {Experiment Management System}
%\item {Faster Training}
%\item {Faster Decoding}
%\item {Moses Server}
%\item {Data and domain adaptation}
%  \begin{itemize} \vspace{-5mm}
%  \item {Train everything together}
%  \item {Secondary phrase table}
%  \item Domain indicator features
%\item Interpolated language models
%  \item \currenttopic{TM-MT integration}
%  \end{itemize}
%\end{itemize}
%}

%%%%%%%%%%%%%%%%%%%%%%%%%%%%%%%%%%%%%%%%%%%%%%%%%%%%%%%%%%%%%%%%%%%%%%%%%%%%

%\slide{TM-MT Integration}
%\begin{itemize}
%\item Input sentence: \vspace{-5mm}
%\begin{center}
%\example{The second paragraph of Article \highlightOrange{21} is deleted .}
%\end{center}
%\item Fuzzy match in translation memory: \vspace{-5mm}
%\begin{center}
%\example{The second paragraph of Article \highlight{5} is deleted .}\\
%=\\
%\example{\highlightGreen{{\`A} l' article} \highlight{5} \highlightGreen{, le texte du deuxi{\'e}me alin{\'e}a est supprim{\'e} .}}
%\end{center}
%\item[] \textcolor{darkgreen}{\bf Output word(s) taken from the target TM}  \vspace{-5mm}
%\item[] \textcolor{darkorange}{\bf Input word(s) that still need to be translated by SMT}
%\end{itemize}


%\slide{TM-MT Integration}

%\begin{itemize} \itemsep -1mm
%\item Translation memory-style fuzzy match
%  \begin{itemize}
%  \item For hierarchical decoding
%  \item Create long translation rule 'templates'
%  \item Best for use with parallel corpus with lots of repetition
%  \end{itemize}

%\item Add TM and word alignment as a special phrase table
%  \begin{itemize}
%    \item Use in addition to normal phrase table
%  \end{itemize}
%  \begin{SaveVerbatim}{myverb} 
%   [feature]
%   PhraseDictionaryFuzzyMatch ...
%   PhraseDictionaryMemory path=phrase-table.gz ...
%   PhraseDictionaryMemory path=glue-rules ...
%  \end{SaveVerbatim}
%  \colorbox{gray}{\BUseVerbatim{myverb}}

%\end{itemize}

%%%%%%%%%%%%%%%%%%%%%%%%%%%%%%%%%%%%%%%%%%%%%%%%%%%%%%%%%%%%%%%%%%%%%%%%%%%%

\slide{Advanced Features}
\vspace{-5mm}
\textcolor{darkgrey}{
\begin{itemize} \itemsep -1mm
\item {Faster Training}
\item {Faster Decoding}
\item {Moses Server}
\item {Domain adaptation}
\item \currenttopic{Transliteration}
\item {Output formats}
\end{itemize}
}

%%%%%%%%%%%%%%%%%%%%%%%%%%%%%%%%%%%%%%%%%%%%%%%%%%%%%%%%%%%%%%%%%%%%%%%%%%%%

\slide{Transliteration}
\begin{itemize}
  \item Languages are written in different scripts 
  \begin{itemize}
    \item Russian, Bulgarian and Serbian - written in Cyrillic script
    \item Urdu, Farsi and Pashto - written in Arabic script
    \item Hindi, Marathi and Nepalese - written in Devanagri
  \end{itemize}
  \item Transliteration can be used to translate OOVs and Named Entities
  \item Problem: Transliteration corpus is not always available
  \item Naive Solution:
  \begin{itemize}
    \item Crawl training data from Wikipedia titles
    \item Build character-based transliteration model
    \item Replace OOV words with 1-best transliteration
  \end{itemize}
\end{itemize}

%%%%%%%%%%%%%%%%%%%%%%%%%%%%%%%%%%%%%%%%%%%%%%%%%%%%%%%%%%%%%%%%%%%%%%%%%%%%

\slide{Transliteration}

\begin{itemize}
  \item 2 methods to integrate into MT
  \item Post-decoding method
  \begin{itemize}  
    \item Use language model to pick best transliteration
    \item Transliteration features
  \end{itemize}
  \item In-decoding method
  \begin{itemize}  
    \item Integrate transliteration inside decoder
    \item Words can be translated OR transliterated
  \end{itemize}
\end{itemize}

%%%%%%%%%%%%%%%%%%%%%%%%%%%%%%%%%%%%%%%%%%%%%%%%%%%%%%%%%%%%%%%%%%%%%%%%%%%%

\slide{Transliteration}

\begin{itemize}
  \item EMS:
  
  \begin{SaveVerbatim}{myverb}
  [TRAINING]
  transliteration-module = "yes"
  \end{SaveVerbatim}
  \colorbox{gray}{\BUseVerbatim{myverb}}
  
  \item Post-processing method

  \begin{SaveVerbatim}{myverb} 
   post-decoding-transliteration = "yes"
  \end{SaveVerbatim}
  \colorbox{gray}{\BUseVerbatim{myverb}}
  
  \item In-decoding method

  \begin{SaveVerbatim}{myverb} 
   in-decoding-transliteration = "yes"
   transliteration-file = /list of words to be transliterated/
  \end{SaveVerbatim}
  \colorbox{gray}{\BUseVerbatim{myverb}}
 
\end{itemize}

%%%%%%%%%%%%%%%%%%%%%%%%%%%%%%%%%%%%%%%%%%%%%%%%%%%%%%%%%%%%%%%%%%%%%%%%%%%%


\slide{Advanced Features}
\vspace{-5mm}
\textcolor{darkgrey}{
\begin{itemize} \itemsep -1mm
\item {Faster Training}
\item {Faster Decoding}
\item {Moses Server}
\item {Domain adaptation}
\item {Transliteration}
\item \currenttopic{Output formats}
\end{itemize}
}


%%%%%%%%%%%%%%%%%%%%%%%%%%%%%%%%%%%%%%%%%%%%%%%%%%%%%%%%%%%%%%%%%%%%%%%%%%%%

\slide{N-Best List}

\begin{itemize}
\item Input \vspace{-5mm}
\begin{center}
\example{es gibt verschiedene andere meinungen .}
\end{center}

\item  Best Translation \vspace{-5mm}
\begin{center}
\example{there are various different opinions .}
\end{center}

\item  Next nine best translations \vspace{-5mm}
{\footnotesize \begin{center}
\example{
there are various other opinions . \\
there are different different opinions . \\
there are other different opinions . \\
we are various different opinions . \\
there are various other opinions of . \\
it is various different opinions . \\
there are different other opinions . \\
it is various other opinions . \\
it is a different opinions .}
\end{center}}
\end{itemize}

%%%%%%%%%%%%%%%%%%%%%%%%%%%%%%%%%%%%%%%%%%%%%%%%%%%%%%%%%%%%%%%%%%%%%%%%%%%%
\slide{Uses of N-Best Lists}
\vspace{20mm}
\begin{itemize}
\item  Let the translator choose from possible translations
\item  Reranker
	\begin{itemize}
	\item add more knowledge sources
	\item can take global view
	\item coherency of whole sentence
	\item coherency of document
	\end{itemize}
\item  Used to tune component weights
\end{itemize}

%n-best translations
%
%1. Best translation for the each input sentence
%2. 10-best translation for each input sentence
%- in sorted orde of best first
%      - think that the decoder can get good translation
%      - but not confident that the decoder will do a good job of finding the best translation.
%      - give 10 translations for the user to decide
%      
%      - in general, can ask the decoder to return the n-best sentences
%      
%      - more often used to give to a post-processing step. 
%      - let another algorithm decide which 1 really is the best sentence. Based on other critieria not in the decoder
%          - document level information.
%          
%          - pronoun translation from chinese/vietnamese to english.
%          - dependent on context.
%             -  no word for 'me'. Could be translated as nephew, uncle, grandfather, friend, depending on who you're talking to
%             -  'you' could be translated as nephew, uncle, grandfather, friend.
%   - external tools which looks at the whole document might do a better job finding the most appropriate. 
%   - give it 100-best translation, let it decide
% 

%%%%%%%%%%%%%%%%%%%%%%%%%%%%%%%%%%%%%%%%%%%%%%%%%%%%%%%%%%%%%%%%%%%%%%%%%%%%
\slide{N-Best Lists in Moses}
\vspace{5mm}
\begin{center}
Argument to command line\\[2mm]
\begin{SaveVerbatim}{myverb}
 ./moses -n-bestlist n-best.file.txt [distinct] 100
\end{SaveVerbatim}
\colorbox{gray}{\BUseVerbatim{myverb}}

\vspace{10mm}
Output\\[2mm]
{\footnotesize \begin{SaveVerbatim}{myverb}
0 ||| there are various different opinions .  ||| d: 0 lm: -21.6664 w: -6 ...  ||| -113.734
0 ||| there are various other opinions .  ||| d: 0 lm: -25.3276 w: -6 ... ||| -114.004
0 ||| there are different different opinions .  ||| d: 0 lm: -27.8429 w: -6 ...  ||| -117.738
0 ||| there are other different opinions .  ||| d: -4 lm: -25.1666 w: -6 ...  ||| -118.007
0 ||| we are various different opinions .  ||| d: 0 lm: -28.1533 w: -6 ...  ||| -118.142
0 ||| there are various other opinions of .  ||| d: 0 lm: -33.7616 w: -7 ...  ||| -118.153
0 ||| it is various different opinions .  ||| d: 0 lm: -29.8191 w: -6  ... ||| -118.222
0 ||| there are different other opinions .  ||| d: 0 lm: -30.426 w: -6 ...  ||| -118.236
0 ||| it is various other opinions .  ||| d: 0 lm: -32.6824 w: -6 ... ||| -118.395
0 ||| it is a different opinions .  ||| d: 0 lm: -20.1611 w: -6 ...  ||| -118.434

\end{SaveVerbatim}
\colorbox{gray}{\BUseVerbatim{myverb}}}
\end{center}
 
%%%%%%%%%%%%%%%%%%%%%%%%%%%%%%%%%%%%%%%%%%%%%%%%%%%%%%%%%%%%%%%%%%%%%%%%%%%%
\slide{Search Graph}

\begin{itemize}
\item  Input \vspace{-10mm}
\begin{center}
\example{er geht ja nicht nach hause}
\end{center}

\item Return internal structure from the decoder \vspace{-5mm}
\begin{center}
\includegraphics[scale=1.2]{search-graph.png}
\end{center}

\item Encode millions of other possible translations\\
(every path through the graph = 1 translation)

\end{itemize}


%Alternative to getting n-best translation.
%Get back internal structure of the decoder.
%
%Directed graph 
%  - left most node is represents a hypothesis that has translated nothing
%  - right most nodes have translated all words
%  - a path from the left to the right is a translation of the input sentence
%  - best translation is 1 of these paths
%  
%Many such paths (millions)
%  - each path is a translation that the decoder has considered

%%%%%%%%%%%%%%%%%%%%%%%%%%%%%%%%%%%%%%%%%%%%%%%%%%%%%%%%%%%%%%%%%%%%%%%%%%%%
\slide{Uses of Search Graphs}
\begin{tabular}{p{9cm}c}
\vspace{-12cm}
\begin{itemize}
\item  Let the translator choose
	\begin{itemize}
	\item Individual words or phrases
	\item 'Suggest' next phrase
	\end{itemize}
\item  Reranker
\item  Used to tune component weights
	\begin{itemize}
	\item More difficult than with n-best list
	\end{itemize}

\end{itemize}

&

\includegraphics[scale=1]{lattice-caitra.png}

\end{tabular}

%%%%%%%%%%%%%%%%%%%%%%%%%%%%%%%%%%%%%%%%%%%%%%%%%%%%%%%%%%%%%%%%%%%%%%%%%%%%
\slide{Search Graphs in Moses}
\vspace{5mm}
\begin{center}
Argument to command line\\[2mm]
\begin{SaveVerbatim}{myverb}
 ./moses -output-search-graph search-graph.file.txt 
\end{SaveVerbatim}
\colorbox{gray}{\BUseVerbatim{myverb}}
\end{center}
\vspace{5mm}

\begin{center}
Argument to command line\\[2mm]
{\small \begin{SaveVerbatim}{myverb}
0 hyp=0 stack=0 forward=36 fscore=-113.734
0 hyp=75 stack=1 back=0 score=-104.943 ... covered=5-5 out=.
0 hyp=72 stack=1 back=0 score=-8.846 ... covered=4-4 out=opinions
0 hyp=73 stack=1 back=0 score=-10.661 ... covered=4-4 out=opinions of
\end{SaveVerbatim}
\colorbox{gray}{\BUseVerbatim{myverb}}

\vspace{-10mm}
\begin{tabular}{p{15cm}}
\begin{itemize} \itemsep -2mm
\item hyp - hypothesis id
\item  stack - how many words have been translated
\item score - total weighted score
\item covered - which words were translated by this hypothesis
\item out - target phrase
\end{itemize}
\end{tabular}}
\end{center}

%%%%%%%%%%%%%%%%%%%%%%%%%%%%%%%%%%%%%%%%%%%%%%%%%%%%%%%%%%%%%%%%%%%%%%%%%%%%
\includepdf[pages={1-16}]{FeatureFunction.pdf}

%%%%%%%%%%%%%%%%%%%%%%%%%%%%%%%%%%%%%%%%%%%%%%%%%%%%%%%%%%%%%%%%%%%%%%%%%%%
\slide{Contribute to Moses}
\vspace{5mm}
\begin{itemize}
\item  Contribute to 
  \begin{itemize}
  \item Moses
  \item GIZA++
  \item MGIZA
  \item KenLM
  \end{itemize}
\item Contribute data
  \item Parallel corpora
  \item Monolingual corpora
\end{itemize}



%%%%%%%%%%%%%%%%%%%%%%%%%%%%%%%%%%%%%%%%%%%%%%%%%%%%%%%%%%%%%%%%%%%%%%%%%%%

\slide{Acknowledgements}
\vspace{5mm}
\begin{table}[h]
\begin{center}
\begin{tabular}{ c  c  c } 

\includegraphics[scale=0.3]{univ-edinburgh.pdf}
&
\includegraphics[scale=0.07]{charles.png}
&
\includegraphics[scale=1]{fbk.png}
\\[1cm]
\includegraphics[scale=0.6]{maryland.png}
&
\includegraphics[scale=1.5]{mit.png}
&
\includegraphics[scale=1]{aachen.png}
\\[1cm]
\includegraphics[scale=0.5]{mosescore-logo-transp.png}
&
&
\includegraphics[scale=1.2]{logo_eubridge_hgwhite.png}
\end{tabular}
\end{center}
\end{table}
\begin{center}
\ldots and many more
\end{center}

%%%%%%%%%%%%%%%%%%%%%%%%%%%%%%%%%%%%%%%%%%%%%%%%%%%%%%%%%%%%%%%%%%%%%%%%%%%%
%\slide{Moses Developers}
%\vspace{10mm}
%\begin{center}
%\footnotesize \rm
%\begin{tabular}{cccc} 
%Abhishek Arun & 
%Adam Lopez & 
%Ales Tamchyna & 
%Alex \\
%Amittai Axelrod &
%Ankit Srivastava &
%Anthony Rousseau &
%Benjamin Gottesman \\ 
%Barry Haddow &
%Ondrej Bojar &
%Chris Callison-Burch & 
%Christine Corbett \\
%Christian Hardmeier &
%Christian Federmann &
%Lane Schwartz &
%David Talbot \\
%Edmund Huber &
%Evan Herbst &
%Andreas Eisele &
%Eva Hasler \\
%Frederic Blain &
%Brooke Cowan &
%Grace M. Ngai&
%Kenneth Heafield \\
%Hieu Hoang &
%H. Leal Fontes &
%Holger Schwenk &
%Josh Schroeder \\
%Jean-Baptiste Fouet &
%Joern Wuebker &
%Jorge Civera &
%Konrad Rawlik \\
%Abby Levenberg &
%Alexandra Birch &
%Bo Fu &
%M.J.Bellino-Machado \\
%Mauro Cettolo &
%Marcello Federico &
%Michael Auli &
%John Joseph Morgan \\
%Mark Fishel &
%Gabriele Antonio Musillo &
%Miles Osborne &
%Nadi Tomeh \\
%Nicola Bertoldi &
%Oliver Wilson &
%Pascual Martinez &
%Philipp Koehn \\
%Phil Williams &
%Bruno Pouliquen &
%Raphael Payen &
%Chris Dyer \\
%Joao Luís Rosas &
%Rico Sennrich &
%Herve Saint-Amand &
%Felipe Sanchez Martinez \\
%Sara Stymne &
%Steven B. Parks &
%Steven Buraje Poggel &
%Andre Lynum \\
%Yizhao Ni &
%David Kolovratnak &
%Sergio Penkale &
%Stephan \\
%Suzy Howlett &
%Wade Shen &
%Yang Gao &
%Tsuyoshi Okita \\
%Alexander Fraser &
%Richard Zens
%
%\end{tabular}
%\end{center}


\begin{comment}
\end{comment}


\end{document}

